\documentclass[mat1]{fmfdelo}
% \documentclass[fin1]{fmfdelo}
% \documentclass[isrm1]{fmfdelo}
% \documentclass[mat2]{fmfdelo}
% \documentclass[fin2]{fmfdelo}
% \documentclass[isrm2]{fmfdelo}

% naslednje ukaze ustrezno napolnite
\avtor{Tom Gornik}

\naslov{Izrek o invarianci odprtih množic}
\title{Domain invariance theorem}

% navedite ime mentorja s polnim nazivom: doc.~dr.~Ime Priimek,
% izr.~prof.~dr.~Ime Priimek, prof.~dr.~Ime Priimek
% uporabite le tisti ukaz/ukaze, ki je/so za vas ustrezni
\mentor{ izr.~prof.~dr.~Jaka Smrekar}
% \mentorica{}
% \somentor{}
% \somentorica{}
% \mentorja{}{}
% \mentorici{}{}

\letnica{2020} % leto diplome

%  V povzetku na kratko opišite vsebinske rezultate dela. Sem ne sodi razlaga organizacije dela --
%  v katerem poglavju/razdelku je kaj, pač pa le opis vsebine.
\povzetek{Zveznost in diskretnost sta v mnogih pogledih dva povsem nasprotujoča si pojma, a izkaže se, da lahko s pomočjo diskretnih množic dokazujemo nekatere lastnosti zveznih funkcij. Predvsem so take množice priročne pri dokazovanju obstoja obstoja posebnih točk, kot sta negibna točka in ničla funkcije. Obstajata dve prednosti takih dokazov, ki ju je vredno izpostaviti. Prva je ta, da si je dogajanje v dokazu dokaj lahko predstavjati in celo narisati (vsaj v primerih, ko dimenzija ni prevelika). Druga prednost je, da so dokazi vsaj do neke mere konstruktivni, kar pomeni, da lahko ocenimo, kje se nahaja posebna točka funkcije. Predstavili bomo uporabo takega načina dokazovanja za dokaz izreka o invarianci odprtih množic.}

%  Prevod slovenskega povzetka v angleščino.
\abstract{In this work we give elementary proof of domain invariance theorem.}

% navedite vsaj eno klasifikacijsko oznako --
% dostopne so na www.ams.org/mathscinet/msc/msc2010.html
\klasifikacija{52A20, 54F45, 54H25}
\kljucnebesede{simpleks, Spernerjeva lema, Poincar\'e-Mirandov izrek, izrek o invarianci odprtih množic, izrek o invarianci dimenzij} % navedite nekaj ključnih pojmov, ki nastopajo v delu
\keywords{simplex, Sperner's lemma, Poincar\'e-Miranda theorem, domain invariance theorem, invariance of dimensions theorem} % angleški prevod ključnih besed

\zapisiMetaPodatke  % poskrbi za metapodatke in veljaven PDF/A-1b standard

% aktivirajte pakete, ki jih potrebujete
\usepackage{tikz, verbatim, subcaption}
\usetikzlibrary{arrows.meta, calc}

\usepackage{bibentry}         % za navajanje literature v programu dela s celim imenom
%\nobibliography{\literatura}
\newcommand{\literatura}{literatura}  % pot do datoteke z literaturo (brez .bib končnice)



% za številske množice uporabite naslednje simbole
\newcommand{\R}{\mathbb R}
\newcommand{\N}{\mathbb N}
\newcommand{\Z}{\mathbb Z}
\newcommand{\C}{\mathbb C}
\newcommand{\Q}{\mathbb Q}


% matematične operatorje deklarirajte kot take, da jih bo Latex pravilno stavil
\DeclareMathOperator{\conv}{conv}
\DeclareMathOperator{\diam}{diam}
\DeclareMathOperator{\Int}{Int}

% vstavite svoje definicije ...
%  \newcommand{}{}
\newcommand{\I}{\mathbb I}
\newcommand{\0}{\underline{0}}
\newcommand{\pA}{\mathcal A}
\newcommand{\pU}{\mathcal U}

\def\citat#1{,,#1''}


\begin{document}
%####################     1. POGLAVJE: UVOD     ####################
\section{Uvod}
Koncept dimenzije prostora se zdi zelo naraven in intuitiven. V svojih zapisih je o dimenzijah govoril že Aristotel. A ko beremo njegova dela, hitro vidimo, da je bila njegova predstava o dimenzijah drugačna od današnje. Aristotel namreč pravi \citat{Premica ima magnitudo v eni smeri, ravnina v dveh smereh, geometrijska telesa pa v treh smereh in poleg teh treh magnitud ni nobene več, kajti te tri so vse} \cite[str.\ 1, moj prevod]{4dim}. Matematiki so dolgo živeli v prepričanju, do so lahko matematični objekti največ tridimenzionalni. Kasneje so se začele pojavljati potrebe po štiridimenzionalnih objektih, saj so bile na primer v mehaniki enačbe veliko lažje, če je eno dimenzijo predstavljal čas. Tako smo počasi prišli do razmišljanja, da imamo večdimenzionalne in celo neskončno dimenzionalne prostore. Problem se pojavi, ko želimo primerjati prostore različnih dimenzij. Intuicija nam pravi, da prostora različnih dimenzij ne moreta biti enaka. V topologiji rečemo, da ne moreta biti homeomorfna. Homeomorfizem je bijektivna zvezna preslikava $f$, katere inverzna preslikava $f^{-1}$ je tudi zvezna. Če lahko prostor $X$ z neko homeomorfno preslikavo preslikamo v prostor $Y$, pravimo, da sta prostora $X$ in $Y$ homeomorfna. Matematiki so že od nekdaj verjeli, da je dimenzija invarianta, kar pomeni, da imata homeomorfna prostora enako dimenzijo. To trditev je bilo zelo težko dokazati ne samo zato, ker je dokaz zapleten, ampak tudi zato, ker ni bilo dobre definicije dimenzije prostora. Kaj sploh je dimenzija prostora? Velikokrat na dimenzijo gledamo kot na najmanjše število parametrov, ki so potrebni za opis nekega prostora. Ta definicija je mnogim matematikom dolgo časa zadoščala, obstajali pa so tudi ti, ki so vanjo dvomili. Eden prvih, ki je izrazil svoje dvome, je bil nemški matematik Georg Cantor v pismu, ki ga je leta 1873 poslal Richardu Dedekindu, prav tako nemškemu matematiku. Čeprav so mnogi matematiki mislili, da je noro poskusiti vsako točko v ravnini izraziti zgolj z enim parametrom iz premice, je Cantor to poskušal doseči in leta 1878 je dokazal, da obstaja bijektivna preslikava -- imenujmo jo Cantorjeva preslikava -- $c : \R \to \R^n$ \cite{Gouvea2011}. To pomeni, da za opis tudi več dimenzionalnih prostorov potrebujemo zgolj en parameter. Kljub presenetljivemu rezultatu, pa obstoj take preslikave ni tako zelo ogrožal intuicije, saj je bila Cantorjeva preslikava močno nezvezna. Kljub temu pa je postalo očitno, da je potrebno poiskati dokaz, kar je neuspešno poskusilo kar nekaj matematikov. Potreba po dokazu se je še bolj pokazala, ko je leta 1890 italijanski matematik Giuseppe Peano predstavil krivulje, ki napolnijo cel prostor. To pomeni, da lahko vsako točko v prostoru $\R^n$ zvezno opišemo samo z enim parametrom $t \in \R$. Problem take preslikave pa je, da ni injektivna, tako da še vedno obstaja upanje, da je intuicija dobra. Potrebno jo je samo posodobiti do trditve, da v primeru dveh prostorov $X$ in $Y$, ki sta različnih dimenzij, ne obstaja zvezna bijektivna preslikava $f : X \to Y$. To je prvi dokazal Brouwer leta 1911. Za dokaz pa je uporabil nekatere topološke rezultate, npr.\ homotopijo, ki jih mnogi ljubiteljski matematiki in tudi študenti dodiplomskega študija matematike ne poznajo. 
V tem delu bomo predstavili elementaren dokaz tega izreka, ki se izogne abstraktnejšim topološkim trditvam. V poglavju~\ref{raz:simpleksi} bomo razvili potrebno besedišče in spoznali nekaj matematičnih objektov, ter njihovih lastnosti, ki so ključni pri dokazih v naslednjih poglavjih.  V razdelkih \ref{raz:PM} in \ref{raz:siritev} uporabimo rezultate iz poglavja~\ref{raz:simpleksi} in dokažemo dva ključna gradnika dokaza Brouwerjevega izreka. Na koncu v poglavju~\ref{raz:ioiom} pa lahko najdemo dokaz izreka o invarianci odprtih množic in izreka o invarianci dimenzije. Vrstni red razdelkov je izbran tako, da ohrani nekaj matematične skrivnosti, na koncu pa se vse skupaj sestavi v celoto. Neučakan bralec pa lahko prebere najprej tudi poglavje~\ref{raz:ioiom} in se kasneje polno motiviran, z vedenjem zakaj in kako je snov uporabna, vrne na začetek ter naknadno zapolni vse vrzeli v dokazu.


%####################     2. POGLAVJE: SIMPLEKSI     ####################
\section{Simpleksi}\label{raz:simpleksi}
V tem razdelku bomo spoznali simplekse in z njimi povezano Spernerjevo lemo. Na koncu pa bomo nekatere lastnosti simpleksov posplošili tudi na kocke. Poimenovanje za simpleks izvira iz latinske besede \citat{simplex}, ki pomeni preprost oziroma enostaven, saj označuje eno najenostavnejših podmnožic prostora $\R^n$.  Simpleks si lahko predstavljamo, kot posplošitev pojma trikotnik, ki je dvodimenzionalen objekt, na ostale dimenzije. Tako v $0$-dimenzionalnem prostoru simpleks označuje točko, enodimenzionalen simpleks nam predstavlja daljico, v dveh dimenzijah dobimo že znani trikotnik, v treh dimenzijah pa simpleks imenujemo tudi tetraeder itn. Na sliki~\ref{fig:simplex} so prikazani simpleksi dimenzij $0$, $1$, $2$, in $3$ \cite{simplex}.
%####################   Simpleksi v različnih dimenzijah    ####################
\begin{figure}[h!]  
	\centering
	\begin{tikzpicture}
% ####################   0 - simplex    ####################
		\filldraw[black] (0, 0) circle (2pt) node[black, below]{$0$};
		\draw (0, -1) node {$0$-simpleks};
% ####################   1 - simplex    ####################
		\filldraw[black] (2, 0) circle (2pt) node[black, below]{$0$};		
		\filldraw[black] (4, 0) circle (2pt) node[black, below]{$1$};
		\draw (2, 0) -- (4, 0);
		\draw (3, -1) node {$1$-simpleks};
% ####################   2 - simplex    ####################
		\filldraw[black] (6, 0) circle (2pt) node[black, below]{$0$};		
		\filldraw[black] (8, 0) circle (2pt) node[black, below]{$1$};
		\filldraw[black] (7, 1.732) circle (2pt) node[black, above]{$2$};
		\draw (6, 0) -- (8, 0);
		\draw (6, 0) -- (7, 1.732);
		\draw (7, 1.732) -- (8, 0);
		\draw (7, -1) node {$2$-simpleks};
% ####################   3 - simplex    ####################
		\filldraw[black] (10, 0) circle (2pt) node[black, below]{$0$};		
		\filldraw[black] (12, 0) circle (2pt) node[black, below]{$1$};
		\filldraw[black] (11.2, 0.35) circle (2pt) node[black, above left = -0.4mm]{$2$};
		\filldraw[black] (11, 1.732) circle (2pt) node[black, above]{$3$};				
		\draw (10, 0) -- (12, 0);
		\draw (10, 0) -- (11, 1.732);
		\draw (11, 1.732) -- (12, 0);
		\draw[dashed] (10, 0) -- (11.2, 0.35);
		\draw[dashed] (11.2, 0.35) -- (11, 1.732);
		\draw[dashed] (11.2, 0.35) -- (12, 0);
		\draw (11, -1) node {$3$-simpleks};
	\end{tikzpicture}
	\caption{Prikazani so simpleksi v dimenzijah $0$, $1$, $2$ in $3$.}\label{fig:simplex}
\end{figure}
Preden podamo natančno definicijo simpleksa, moramo spoznati afino neodvisne množice. Pri tem se bomo držali pravila, da bomo krajevni vektor od koordinatnega izhodišča do neke točke označili z enako oznako kot točko samo, le da bomo nad oznako narisali puščico. Torej krajevni vektor neke točke $x \in \R^n$ bomo označili z $\vec{x}$.
\begin{definicija}\protect{\cite{simplex}}
Množica točk $x_0, x_1, \dots , x_n \in \R^n$ je \emph{afino neodvisna}, če je množica vektorjev $\vec{x}_1 - \vec{x}_0, \vec{x}_2 - \vec{x}_0, \dots , \vec{x}_n - \vec{x}_0$ linearno neodvisna. V nasprotnem primeru je množica točk \emph{afino odvisna}.
\end{definicija}

Opazimo lahko, da ima v tej definiciji prva točka, tj.\ $x_0$, posebno vlogo. Ker ne govorimo o urejenih množicah in vrstni red elementov ni pomemben, se moramo prepričati, da nam definicija res karakterizira afino neodvisne množice in ni odvisna od izbire prve točke. Denimo, da je množica vektorjev $\vec{x}_1 - \vec{x}_0, \vec{x}_2 - \vec{x}_0, \dots , \vec{x}_n - \vec{x}_0$ linearno neodvisna. Potem je enakost $\sum\limits_{i=1}^n \alpha_i (\vec{x}_i - \vec{x}_0) = 0$ izpolnjena natanko tedaj, ko so vsi koeficienti enaki $0$, torej velja $\alpha_i = 0$ za vsak $i \in \{ 1, 2, \dots, n \}$. Recimo, da smo za prvi element izbrali neko drugo točko, na primer $x_k$. Radi bi videli, da je potem tudi enakost $\sum\limits_{\substack{i=0 \\ i\neq k}}^n \beta_i (\vec{x}_i - \vec{x}_k) = 0$ izpolnjena samo v primeru, ko so vsi koeficienti v vsoti enaki $0$. V to se prepričamo tako, da uporabimo znan matematični trik ter odštejemo in prištejemo isto vrednost. V tem primeru odštejemo in prištejemo $x_0$.
Računamo:
\begin{align*} 
\sum\limits_{\substack{i=0 \\ i \neq k}}^n \beta_i (\vec{x}_i - \vec{x}_k) &=  \sum\limits_{\substack{i=0 \\ i \neq k}}^n \beta_i (\vec{x}_i - \vec{x}_0 + \vec{x}_0 - \vec{x}_k) = \\
&=\sum\limits_{\substack{i=0 \\ i\neq k}}^n \beta_i (\vec{x}_i - \vec{x}_0) + \sum\limits_{\substack{i=0 \\ i\neq k}}^n \beta_i (\vec{x}_0 - \vec{x}_k) \\
&= \sum\limits_{i=1}^n \gamma_i (\vec{x}_i - \vec{x}_0),
\end{align*}
kjer koeficiente $\gamma_i$ določimo na naslednji način:
\[  \gamma_i =  \left\{
\begin{array}{cl}\vspace{3pt}
	-\sum\limits_{\substack{i=0 \\ i\neq k}}^n \beta_i &, i = k,\\
	\beta_i &, i \neq k. \\
\end{array} 
\right. \]
Ker je množica vektorjev $\vec{x}_1 - \vec{x}_0, \vec{x}_2 - \vec{x}_0, \dots , \vec{x}_n - \vec{x}_0$ linearno neodvisna, morajo biti vsi koeficienti $\gamma_i = 0$ za vsak $i \in \{1, 2, \dots, n \}$. Torej so tudi vsi koeficienti $\beta_i = 0$ za vsak $i \in \{0, 1, \dots, k-1, k+1, \dots, n \}$. Pokazali smo linearno neodvisnost množice $\vec{x}_0 - \vec{x}_k, \vec{x}_1 - \vec{x}_k, \dots , \vec{x}_{k-1} - \vec{x}_k, \vec{x}_{k+1} - \vec{x}_k, \dots, \vec{x}_n - \vec{x}_k$, zato je zgornja definicija dobra. 

S pomočjo točk v prostoru lahko definiramo različne množice. Mi si bomo želeli take množice, ki imajo naslednjo lastnost.
\begin{definicija}
Poljubna množica $K \in \R^n$ je \emph{konveksna}, če je za vsaki dve točki $x, y \in K$ tudi daljica določena z $x$ in $y$: 
$\left \{ \lambda \vec{x} + (1 - \lambda) \vec{y}; \lambda \in  [0, 1] \right \}$ 
cela vsebovana v $K$. Če množoca ni konveksna, pravimo, da je \emph{konkavna}. Primer konveksne in konkavne množice si lahko pogledamo na sliki~\ref{fig:konveksnost}.
% ####################   Primer in protiprimer konveksne množice    ####################
\begin{figure}[h]  
\centering 
% ####################   Primer konveksne množice    ####################
	\begin{subfigure}[b]{0.4\linewidth} 
		\begin{tikzpicture}[scale=0.8]
		\draw[white] (-4, 0) circle (2pt);
		\draw plot [smooth cycle, tension = 1] coordinates {(-2.5, 0.3) (-1.3, -1.8) (1.5, 0.5)};
		\filldraw (0.5, 0.3) circle (2pt);
		\filldraw (-1, -1.3) circle (2pt);
		\draw[line width=0.6pt] (0.5, 0.3) -- (-1, -1.3);
		\draw (0.1, 0.3) node {$x$};
		\draw (-1.3, -1.3) node {$y$};
		\end{tikzpicture}%
		\caption{Množica $A$ je konveksna, saj je vsaka daljica, ki povezuje poljubni točki iz $A$ cela vsebovana v $A$.}\label{fig:konv}
	\end{subfigure}
	\hspace{1cm}
% ####################   Primer konkavne množice    ####################
	\begin{subfigure}[b]{0.4\linewidth}
		\begin{tikzpicture}[scale=0.9]
		\draw plot [smooth cycle, tension = 1] coordinates {(1, 1.3) (-2, 0) (0.8, -1.5) (0, 0)};
		\begin{scope}
			\clip plot [smooth cycle, tension = 1] coordinates {(1, 1.3) (-2, 0) (0.8, -1.5) (0, 0)};
			\draw[line width= 0.6pt]  (0.5, 1) -- (0.2, -1.2);
		\end{scope}
		\draw[red, line width=0.6pt] ($($(0.5, 1)!.5!((0.2, -1.2)$)!23.8pt!(0.5, 1)$) -- ($($(0.5, 1)!.5!((0.2, -1.2)$)!16.9pt!((0.2, -1.2)$);
		\filldraw (0.5, 1) circle (2pt);
		\draw[white] (-3, 0) circle (2pt);
		\filldraw (0.2, -1.2) circle (2pt);
		\draw (0.2, 0.9) node {$x$};
		\draw (0, -1) node {$y$};
		\end{tikzpicture}
		\caption{Množica $B$ je konkavna, saj  del daljice določene s točkama $x$ in $y$, ki je pobarvan rdeče, ni vsebovan v $B$.} \label{fig:nikonv}  
	\end{subfigure}
\caption{Slika prikazuje primer konveksne množice (levo) in konkavne množice (desno).}\label{fig:konveksnost} 
\end{figure} 
\end{definicija}
Konveksnih množic, ki vsebujejo dane točke $x_0, x_1, \dots, x_n$, je veliko. Poglejmo si, kako konstruiramo eno izmed njih.
\begin{definicija}\protect{\cite{convexhull}}
Za poljubne točke $x_0, x_1, \dots, x_n$ v prostoru $\R^n$ definiramo \emph{konveksno ogrinjačo} teh točk $A = \conv(x_0, x_1, \dots, x_n)$ na naslednji način:
$$A = \left \{ \sum\limits_{i=0}^n \lambda_i \vec{x}_i; \text{ kjer so } \lambda_i \in [0, 1] \text{ za vsak } i \in \{0, 1, \dots, n \} \text{ in } \sum\limits_{i=0}^n \lambda_i = 1  \right \}$$
\end{definicija}

Krajevni vektor vsake točke $x \in A = \conv (x_0, \dots, x_n)$ lahko izrazimo kot vsoto $\vec{x} = \sum\limits_{i=0}^n \lambda_i \vec{x}_i$, ki zadošča pogoju $\sum\limits_{i=0}^n \lambda_i = 1$, kjer so vsi $\lambda_i$ pozitivna realna števila. Vsoto, ki zadošča zahtevanim pogojem, imenujemo \emph{konveksna kombinacija} točk $x_0, x_1, \dots , x_n$. Konveksno  ogrinjačo točk $x_0, x_1, \dots, x_n$ si v dveh dimenzijah si lahko predstavljamo kot množico, ki jo omeji elastika, ko jo napnemo čez vse točke in nato spustimo. Prikaz konveksne ogrinjače si lahko pogledamo na sliki~\ref{fig:konvexhull}. Zakaj smo izbrali ravno to množico, nam pojasni trditev~\ref{trd:min-conv}.

% ####################   Konveksna množica z elastiko    ####################
\begin{figure}[h]  
\centering 
	\begin{tikzpicture}
% ####################   elastika    ####################
		\draw plot [smooth cycle, tension = 1] coordinates {(-1, 2.2) (-2.5, 0) (-1.3, -1.8) (1.3, -1.7) (2.3, 0.9)};
% ####################   puščice, ki označujejo, da se elastika skrči    ####################
		\draw[->] (-2, 1.6) -- (-1.8, 1.4);
		\draw[->] (-2.25, -1.05) -- (-2, -0.9);
		\draw[->] (0, -2.1) -- (0, -1.8);
		\draw[->] (2.5, 0) -- (2.2, 0);
		\draw[->] (0.9, 1.9) -- (0.75, 1.65);
% ####################   robne točke, ki jih omejuje elastika    ####################
		\filldraw[black] (-1, 1.7) circle (1pt);
		\filldraw[black] (-1.3, -0.9) circle (1pt);
		\filldraw[black] (2, 1) circle (1pt);
		\filldraw[black] (-2, 0.5) circle (1pt);
		\filldraw[black] (0.5, -1.5) circle (1pt);
% ####################   notranje točke, ki jih omejuje elastika    ####################
		\filldraw[black] (-0.7, 0.3) circle (1pt);
		\filldraw[black] (0.3, 1) circle (1pt);
		\filldraw[black] (0, -0.2) circle (1pt);
		\filldraw[black] (0.8, 0.1) circle (1pt);
% ####################   oblika elastike, ko jo izpustimo    ####################
		\filldraw[fill=gray!20] (7, 1.7) -- (6, 0.5)  -- (6.7, -0.9)  -- (8.5, -1.5) -- (10, 1) -- cycle;
% ####################   robne točke, na katerih je napeta elastika    ####################
		\filldraw (7, 1.7) circle (1pt);
		\filldraw (6.7, -0.9) circle (1pt);
		\filldraw (10, 1) circle (1pt);
		\filldraw (6, 0.5) circle (1pt);
		\filldraw (8.5, -1.5) circle (1pt);
% ####################   notranje točke    ####################
		\filldraw[black] (7.3, 0.3) circle (1pt);
		\filldraw[black] (8.3, 1) circle (1pt);
		\filldraw[black] (8, -0.2) circle (1pt);
		\filldraw[black] (8.8, 0.1) circle (1pt);
% ####################   puščice    ####################
		\draw[->] (3, 1) -- (5.5,1);
		\draw (4.25, 1.8) node {Izpustimo};
		\draw (4.25, 1.4) node {elastiko!};
	\end{tikzpicture}
	\caption{Slika prikazuje, kako dobimo konveksno ogrinjačo točk s pomočjo elastike. Na levi strani napnemo elastiko, tako da zaobjame vse točke. Na desni strani pa je prikazano, kako izgleda elastika po tem, ko jo izpustimo in nam omeji konveksno ogrinjačo -- sivo pobarvana množica -- prikazanih točk \protect{\cite{convexhull}}.} \label{fig:konvexhull}
\end{figure} 

\begin{trditev}\protect{\cite{convexhull}}\label{trd:min-conv}
Množica $A = \conv(x_0, x_1, \dots, x_n)$ je najmanjša konveksna množica, ki vsebuje točke $x_0, x_1, \dots, x_n$.
\end{trditev}
\begin{dokaz}
Najprej bomo pokazali, da je pri nekih danih točkah $x_0, x_1, \dots, x_n$ množica $A = \conv(x_0, x_1, \dots, x_n)$ konveksna, nato se bomo prepričali, da je najmanjša med vsemi konveksnimi množicami, ki vsebujejo točke $x_0, x_1, \dots, x_n$. Vzemimo dve točki s krajevnima vektorjema $\vec{x} = \sum\limits_{i=0}^n \lambda_i \vec{x}_i$ in $\vec{y} = \sum\limits_{i=0}^n \mu_i \vec{x}_i$ iz množice $A$, kjer sta vsoti konveksni kombinaciji točk $x_0, x_1, \dots, x_n$.
Krajevni vektor poljubne točke $z$ na daljici določeni z $x$ in $y$ lahko zapišemo kot:
$$\vec{z} = t \vec{x} + (1-t) \vec{y} = t \sum\limits_{i=0}^n \lambda_i \vec{x}_i + (1 - t) \sum\limits_{i=0}^n \mu_i \vec{x}_i = \sum\limits_{i=0}^n (t \lambda_i + (1 - t) \mu_i) \vec{x}_i,$$
kjer je $t \in [0, 1]$. Ker so koeficienti $\lambda_i$ in $\mu_i$ pozitivni za vsak $i = 0, 1, \dots n$ velja neenakost $t \lambda_i + (1 - t) \mu_i \geq 0$. Ker velja tudi
$$\sum\limits_{i=0}^n (t \lambda_i + (1 - t) \mu_i) = t \sum\limits_{i=0}^n \lambda_i + (1 - t) \sum\limits_{i=0}^n \mu_i = t \cdot 1 + (1 - t) \cdot 1 = 1,$$
je $z \in A$.

Dokazati moramo še, da je $A$ najmanjša konveksna množica, ki vsebuje točke $x_0, x_1, \dots, x_n$. Denimo, da odstranimo eno točko $x$ iz množice $A$ s krajevnim vektorjem $\vec{x} = \sum\limits_{i=0}^n \lambda_i \vec{x}_i$. Da bo ta množica $A \setminus \{ x \}$ še vedno vsebovala točke $x_0, x_1, \dots, x_n$, to ne sme biti ena izmed njih. Zato sta vsaj dva koeficienta $\lambda_i$ v izražavi točke $x$ neničelna. Brez izgube splošnosti lahko predpostavimo, da je $\lambda_0 \neq 0$. Definirajmo $\lambda =  \sum\limits_{i=1}^n \lambda_i$. Sedaj lahko krajevni vektor točke $x$ izrazimo malo drugače:
$$\vec{x} = \sum\limits_{i=0}^n \lambda_i \vec{x}_i = \lambda_0 \vec{x}_0 + \sum\limits_{i=1}^n \lambda_i \vec{x}_i = \lambda_0 \vec{x}_0 + \lambda \sum\limits_{i=1}^n \frac{\lambda_i}{\lambda} \vec{x}_i$$
Torej točka $x$ leži na premici določeni s točkama $x_0$ in $\sum\limits_{i=1}^n \frac{\lambda_i}{\lambda} \vec{x}_i$. Ker so koeficienti $\frac{\lambda_i}{\lambda} \geq 0$ in je $\sum\limits_{i=1}^n \frac{\lambda_i}{\lambda} = 1$, obe točki ležita v množici $A \setminus \{ x \}$. Dobili smo taki točki iz $A \setminus \{ x \}$, pri katerih ne velja, da je celotna daljica, določena z njima, vsebovana v množici $A \setminus \{ x \}$, kar pomeni, da množica $A \setminus \{ x \}$ ni konveksna. Torej je $A$ res najmanjša konveksna množica, ki vsebuje točke $x_0, x_1, \dots, x_n$.
\end{dokaz}
Simpleks definiramo kot posebno konveksno ogrinjačo.
\begin{definicija}\protect{\cite{convexhull}}
Naj bo $V = \{x_0, x_1, \dots , x_n \} \subset \R^n$ afino neodvisna množica točk v Evklidskem prostoru $\R^n$. Konveksni ogrinjači $S$ točk $x_0, x_1, \dots , x_n$ iz množice $V$ pravimo \emph{$n$-dimenzionalni simpleks} ali \emph{$n$-simpleks}. Točke $x_i$, imenujemo \emph{oglišča} simpleksa $S$. Ko želimo poudariti, katera oglišča določajo simpleks, lahko zapišemo tudi $S = \left < x_0, \dots, x_n \right >$. Za vsako neprazno podmnožico $U = \{ y_0, y_1, \dots, y_r \} \subset V$ lahko definiramo simpleks $L = \langle y_0, y_1, \dots, y_r \rangle$, ki ga imenujemo \emph{lice simpleksa} $S$. Če je lice $L$ $(n-1)$-simpleks, ga imenujemo \emph{pravo lice} simpleksa $S$. 
\end{definicija}
 Krajevni vektor vsake točke $x \in S = \left < x_0, \dots, x_n \right >$ lahko izrazimo s pomočjo oglišč kot vsoto:
\begin{equation}\label{eqn:tockasimpleksa}
\vec{x} = \sum\limits_{i=0}^n \lambda_i \vec{x}_i
\end{equation}
ki zadošča pogoju $\sum\limits_{i=0}^n \lambda_i = 1$, kjer so vsi $\lambda_i$ pozitivna realna števila. Zakaj smo konveksni ogrinjači dodali pogoj o afini neodvisnosti točk bomo utemeljili z naslednjim razmislikom. Najprej preoblikujemo izraz~\eqref{eqn:tockasimpleksa}. 
\begin{align*}
\vec{x} &= \sum\limits_{i=0}^n \lambda_i \vec{x}_i =\\
&= \sum\limits_{i=0}^n \lambda_i (\vec{x}_i - \vec{x}_0 + \vec{x}_0) =\\
&= \sum\limits_{i=0}^n \lambda_i \vec{x}_x + \sum\limits_{i=1}^n \lambda_i (\vec{x}_i - \vec{x}_0) =\\
&= \vec{x}_0 + \sum\limits_{i=1}^n \lambda_i (\vec{x}_i - \vec{x}_0).
\end{align*}
Torej laho simpleks $S = \langle x_0, x_1, \dots, x_n \rangle$ zapišemo tudi v obliki:
$$S = \vec{x}_0 + \underbrace{\left \{ \sum\limits_{i=1}^n \lambda_i (\vec{x}_i - \vec{x}_0); \lambda_i \in [0, 1] \text{ za vsak } i \in \{0, 1, \dots, n \} \text{ in } \sum\limits_{i=0}^n \lambda_i = 1  \right \}}_{\Delta}.$$
Ker so vektorji $\vec{x}_1 - \vec{x}_0, \vec{x}_2 - \vec{x}_0, \dots, \vec{x}_n - \vec{x}_0$ linearno neodvisni, je množica $\Delta$ $n$-dimenzionalna podmnožica prostora $\R^n$. Simpleks $S$ je enak množici $\Delta$ premaknjeni za vektor $\vec{x}_0$, zato je tudi simpleks $S$ $n$-dimenzionalna množica. Če bi bile točke $x_0, x_1, \dots, x_n$ afino odvisne, bi bili vektorji $\vec{x}_1 - \vec{x}_0, \vec{x}_2 - \vec{x}_0, \dots, \vec{x}_n - \vec{x}_0$ linearno odvisni in bi bila množica $\Delta$ največ $(n-1)$-dimenzionalna. Tako z vsakim dodatnim ogliščem povečamo dimenzijo simpleksa.
Števila $\lambda_i$ v enačbi~\eqref{eqn:tockasimpleksa} so \emph{baricentrične koordinate} točke $x$, kar zapišemo kot $x = (\lambda_0, \dots, \lambda_n)_b$.
\begin{trditev}\label{trd:zveznost-baricentra}
Baricentrične koordinate poljubne točke $x \in \left <x_0, \dots, x_n \right >$ so zvezne funkcije točke $x$ in velja $x = (\lambda_0(x), \dots, \lambda_n(x))_b$.
\end{trditev}
\begin{dokaz}
Najprej bomo pokazali, da so baricentrične koordinate enolično določene s točko $x$, nato bomo utemeljili še zvezno odvisnost od točke $x$. Recimo, da $x$ izrazimo na dva načina kot $x = \left (\alpha_0, \dots, \alpha_n \right )_b = \left (\beta_0, \dots, \beta_n \right )_b$. Potem lahko zapišemo 
$$\sum_{i=0}^n \alpha_i \vec{x}_i = \sum_{i=0}^n \beta_i \vec{x}_i.$$
Če enačbi odštejemo desno stran, dobimo 
$$\sum_{i=0}^n \alpha_i \vec{x}_i - \sum_{i=0}^n \beta_i \vec{x}_i = 0,$$
kar preoblikujemo v enačbo
$$\sum_{i=0}^n (\alpha_i - \beta_i) \vec{x}_i  = 0.$$
Uporabimo že znan trik ter odštejemo in prištejemo isto vrednost,
$$\sum_{i=0}^n (\alpha_i  - \beta_i) \cdot (\vec{x}_i - \vec{x}_0 + \vec{x}_0) = 0.$$
Enačbo lahko zapišemo z dvema vsotama
$$\sum_{i=1}^n (\alpha_i  - \beta_i) \cdot (\vec{x}_i -\vec{x}_0) + \sum_{i=0}^n (\alpha_i  - \beta_i) \cdot \vec{x}_0= 0,$$
kjer pri prvi vsoti parameter $i$ začne teči pri $1$, saj je pri vrednosti $0$ tudi vrednost sumanda  $(\alpha_0 - \beta_0) \cdot (\vec{x}_0 -\vec{x}_0)$ enaka $0$. Druga vsota je enaka $0$, saj velja:
$$\sum_{i=0}^n (\alpha_i  - \beta_i) \cdot \vec{x}_0= \vec{x}_0 \left (\sum_{i=0}^n \alpha_i  - \sum_{i=0}^n \beta_i \right) = \vec{x}_0 (1 - 1)= 0.$$
Tako pridemo do enačbe 
$$\sum_{i=1}^n (\alpha_i  - \beta_i) \cdot (\vec{x}_i - \vec{x}_0).$$
Ker so vektorji $(\vec{x}_i - \vec{x}_0)$ linearno neidvisni za vsak $i \in \{ 1, \dots, n \}$ morajo biti vsi koeficienti $\alpha_i  - \beta_i = 0$, zato velja $\alpha_i  = \beta_i$ za vsak $i \in \{ 1, \dots, n \}$. Sedaj lahko iz enakosti
$$\sum_{i=0}^n \alpha_i \vec{x}_i = \sum_{i=0}^n \beta_i \vec{x}_i$$
sklepamo, da sta enaka tudi koeficienta $\alpha_0$ in $\beta_0$.

Za dokaz zveznosti baricentričnih koordinat opazimo, da je množica vektorjev $B = \{ \vec{x}_1 - \vec{x}_0, \vec{x}_2 - \vec{x}_0, \dots, \vec{x}_{n} - \vec{x}_0 \}$ baza prostora $\R^n$. Koordinate vektorja od točke $x_0$, do poljubne točke $a \in S$ v bazi $B$ so zvezno odvisne od  točke $a$. Razvijemo vektor $\vec{a} - \vec{x}_0$ po bazi $B$: $\vec{a} - \vec{x}_0 = \sum\limits_{i=1}^n \alpha_i (\vec{x}_i - \vec{x}_0)$, kjer so za vsak $i = 1, 2, \dots, n$ koeficienti zvezne funkcije točke $a$: $\alpha_i = \alpha_i(a)$. Ker je točka $a$ znotraj simpleksa, so koeficienti $\alpha_i \in [0, 1]$ in velja $\sum\limits_{i=1}^n \alpha_i \leq 1$. Potem lahko vektor $\vec{a}$ zapišemo kot $\vec{a}= \vec{x}_0 + \sum\limits_{i=1}^n \alpha_i (\vec{x}_i - \vec{x}_0)$ Vsoto lahko preoblikujemo:
$$\vec{a}= \vec{x}_0 +  \sum_{i=1}^n \alpha_i (\vec{x}_i - \vec{x}_0) = \sum_{i=1}^n \alpha_i \vec{x}_i +(1 - \sum_{i=1}^n \alpha_i ) \vec{x}_0 = \sum_{i=0}^n \gamma_i \vec{x}_i,$$
kjer je
\[  \gamma_i =  \left\{
\begin{array}{cl}\vspace{3pt}
	1 -\sum\limits_{j=1}^n \alpha_j & i = 0,\\
	\alpha_i & i \neq 0. \\
\end{array} 
\right. \]
Za koeficiente $\gamma_i$ velja, da $0 \leq \gamma_i \leq 1$ in $\sum\limits_{i=0}^n \gamma_i = 1$. To pomeni, da točko $a$ lahko zapišemo s pomočjo baricentričnih koordinat $a = (\gamma_0, \gamma_1, \dots, \gamma_n)_b$. Ker so koeficienti $\alpha_i = \alpha_i(a)$ zvezne funkcije za vsak $i = 1, 2, \dots, n$, so zvezne funkcije tudi koeficienti $\gamma_j = \gamma_j(a)$ za vsak $j = 0, 1, \dots, n$.
\end{dokaz}
Pri dokazovanju različnih rezultatov s pomočjo simpleksov si večkrat pomagamo z delitvijo simpleksa na manjše simplekse. Da imamo čim več nadzora pri dokazovanju, definiramo posebno delitev simpleksa, ki jo imenujemo triangulacija.

\begin{definicija}
\emph{Triangulacija} $T$ $n$-simpleksa $S$ je vsaka množica simpleksov, za katero veljata naslednji lastnosti:
\begin{enumerate}
\item Unija vseh simpleksov iz $T$ enaka $S$.
\item Neprazen presek $L$ dveh simpleksov iz $T$ je lice obeh simpleksov in velja $L \in T$.
\end{enumerate}
Ogliščem poljubnega simpleksa $R \in T$ pravimo \emph{vozlišča} triangulacije $T$.
\end{definicija}

% ####################   Primer in protiprimer triangulacije    ####################
\begin{figure}[h]  
\centering 
% ####################   Primer triangulacije    ####################
	\begin{subfigure}[b]{0.4\linewidth} 
		\begin{tikzpicture}[scale=0.9]
			\filldraw[black] (0, 0) circle (1pt);
			\filldraw[black] (2, 1.5) circle (1pt);
			\filldraw[black] (1.7, {1.7*sqrt(3)}) circle (1pt);
			\filldraw[black] (3.5, {2.5*sqrt(3)}) circle (1pt);
			\filldraw[black] (3, 2.25) circle (1pt);
			\filldraw[black] (5, 0) circle (1pt);
			\filldraw[black] (6, 0) circle (1pt);
			\filldraw[black] (3, {3*sqrt(3)}) circle (1pt);
			\filldraw[black] (5, {sqrt(3)}) circle (1pt);
			\filldraw[black] (0.5, {0.5*sqrt(3)}) circle (1pt);
			\filldraw[black] (4.3, 1) circle (1pt);
			\filldraw[black] (1, {1*sqrt(3)}) circle (1pt);
			\draw (0, 0) -- (6, 0);
			\draw (0, 0) -- (3, 2.25);
			\draw (6, 0) -- (3, {3*sqrt(3)});
			\draw (3, {3*sqrt(3)}) -- (0, 0);
			\draw (4.3, 1) -- (6, 0);
			\draw (0, 0) -- (4.3, 1);
			\draw (4.3, 1) -- (5, {sqrt(3)});
			\draw (4.3, 1) -- (5, 0);
			\draw (4.3, 1) -- (3.5, {2.5*sqrt(3)});
			\draw (4.3, 1) -- (2, 1.5);
			\draw (4.3, 1) -- (3, 2.25);
			\draw (1.7, {1.7*sqrt(3)}) -- (3, 2.25);
			\draw (2, 1.5) -- (0.5, {0.5*sqrt(3)});
			\draw (2, 1.5) -- (1.7, {1.7*sqrt(3)});
			\draw (3.5, {2.5*sqrt(3)}) -- (3, 2.25);
			\draw (3.5, {2.5*sqrt(3)}) -- (1.7, {1.7*sqrt(3)});
			\draw (2, 1.5) -- (1, {1*sqrt(3)});
		\end{tikzpicture}%
		\caption{Delitev je triangulacia.} \label{fig:triang}
	\end{subfigure}
	\hspace{1cm}
% ####################   Primer delitve, ki ni triangulacija    ####################
	\begin{subfigure}[b]{0.4\linewidth}
		\begin{tikzpicture}[scale=0.9]

			\fill[fill=gray!40] (0, 0) -- (4.3, 1) -- (3, 2.25);
			\fill[fill=gray!40] (1.7, {1.7*sqrt(3)}) -- (2, 1.5) -- (3, 2.25);
			\filldraw[black] (0, 0) circle (1pt);
			\filldraw[black] (2, 1.5) circle (1pt);
			\filldraw[black] (1.7, {1.7*sqrt(3)}) circle (1pt);
			\filldraw[black] (3.5, {2.5*sqrt(3)}) circle (1pt);
			\filldraw[black] (3, 2.25) circle (1pt);
			\filldraw[black] (5, 0) circle (1pt);
			\filldraw[black] (6, 0) circle (1pt);
			\filldraw[black] (3, {3*sqrt(3)}) circle (1pt);
			\filldraw[black] (5, {sqrt(3)}) circle (1pt);
			\filldraw[black] (0.5, {0.5*sqrt(3)}) circle (1pt);
			\filldraw[black] (4.3, 1) circle (1pt);
			\filldraw[black] (1, {1*sqrt(3)}) circle (1pt);
			\draw (0, 0) -- (6, 0);
			\draw (0, 0) -- (3, 2.25);
			\draw (6, 0) -- (3, {3*sqrt(3)});
			\draw (3, {3*sqrt(3)}) -- (0, 0);
			\draw (4.3, 1) -- (6, 0);
			\draw (0, 0) -- (4.3, 1);
			\draw (4.3, 1) -- (5, {sqrt(3)});
			\draw (4.3, 1) -- (5, 0);
			\draw (4.3, 1) -- (3.5, {2.5*sqrt(3)});
			\draw (4.3, 1) -- (3, 2.25);
			\draw (1.7, {1.7*sqrt(3)}) -- (3, 2.25);
			\draw (2, 1.5) -- (0.5, {0.5*sqrt(3)});
			\draw (2, 1.5) -- (1.7, {1.7*sqrt(3)});
			\draw (3.5, {2.5*sqrt(3)}) -- (3, 2.25);
			\draw (3.5, {2.5*sqrt(3)}) -- (1.7, {1.7*sqrt(3)});
			\draw (2, 1.5) -- (1, {1*sqrt(3)});
		\end{tikzpicture}
		\caption{Delitev ni triangulacija.} \label{fig:nitriang}  
	\end{subfigure}
\caption{Slika prikazuje primer triangulacije (levo) in primer delitve, ki ni triangulacija (desno), saj presek osenčenih simpleksov ni lice obeh simpleksov.}
\end{figure} 

Včasih imamo zaradi lažje predstave raje bolj simetrične triangulacije. Takrat simpleks razdelimo s skladnimi simpleksi.

\begin{definicija}
Kadar je simpleks $S$ razdeljen s skladnimi simpleksi, govorimo o \emph{baricentrični triangulaciji}. Za vsako naravno število $k \in \N$ lahko definiramo triangulacijo, v kateri je množica vozlišč enaka množici $V_k$ definirani na naslednji način: 
$$V_k = \left\{\sum\limits_{i=0}^n \frac{\lambda_i}{k} x_i, \text{ kjer je } \sum\limits_{i=0}^n \lambda_i = k \text{ in } \lambda_i \in \N_0 \text{ za vsak } i = 0, 1, \dots, n \right\}.$$
Taki triangulaciji pravimo \emph{$k$-baricentrična triangulacija} simpleksa.
\end{definicija}

Pomembna prednost baricentričnih triangulacij je ta, da imamo nadzor nad tem, kako veliki simpleksi nastopajo v triangulaciji. Velikost neke množice $A \in \R^n$ je mogoče definirati na več različnih načinov. Da se izognemo zmedi, bomo za velikost množice uporabljali besedo diameter, ki  jo definiramo na naslednji način.
\begin{definicija}
\emph{Diameter} neprazne množice $A \subset \R^n$ označimo z $\diam(A)$ in je definiran kot $\diam(A) = \sup \left \{ d(x, y) = \| x - y \|; x, y \in A \right \}$.
\end{definicija}
Če imamo podan $n$-simpleks $S$ in $k$-baricentrično triangulacijo $T$, potem je za vsak $n$-simpleks $R \in T$ diameter $\diam(R) = \frac{\diam(S)}{k}$. Pomembna posledica te ugotovitve je, da gredo diametri simpleksov v $k$-baricentrični triangulaciji proti $0$, ko gre $k$ proti neskončno.


\begin{definicija}~\protect{\cite[str.\ 9, definicija 1]{Ahlbach}}
Naj bo podan $n$-simpleks $S \in \R^n$ s triangulacijo $T$. Označimo množico vozlišč triangulacije $T$ z $V$. Preslikavo $b : V \to \left \{0, 1, \dots, n \right \} $ imenujemo \emph{barvanje triangulacije} $T$. Za vozlišče $v \in V$ številu $b(v)$ pravimo \emph{barva} ali \emph{oznaka} vozlišča $v$.
\end{definicija}
Barve navadno označujemo s številkami, saj bi pri velikem številu različnih barv nastala zmeda in bi se nekatere barve težko ločilo med seboj. Kadar v triangulaciji nastopa majhno število barv, so slike lepše, če jih ponazorimo s pravimi barvami. Če določimo pravila, s kakšnimi barvami lahko pobarvamo določena oglišča, lahko dobimo zanimive lastnosti. Primer takega pravila in njegove lastnosti nam je opazoval in opisal Emanuel Sperner, po komer se imenuje eno od pravil, za barvanje triangulacije.
\begin{definicija}~\protect{\cite[str.\ 9, definicija 2]{Ahlbach}}
Kadar je vsako oglišče simpleksa $S$ pobarvano s svojo barvo, vozlišča triangulacije $T$ vsebovana v nekem licu $L$ simpleksa $S$, imajo enako oznako kot eno od oglišč, ki to lice določajo, govorimo o \emph{Spernerjevem barvanju}.
\end{definicija}
\newcommand*\rows{6}
\newcommand*\vel{1.8}
%####################   Spernerjevo barvanje    ####################
\begin{figure}[h!]    
	\centering
	\begin{tikzpicture}
		\fill[fill=gray!25] ($\vel*(3.5, {0.5*sqrt(3)})$)--($\vel*(3, {sqrt(3)})$)--($\vel*(4, {sqrt(3)})$);
		\fill[fill=gray!25] ($\vel*(3, {sqrt(3)})$)--($\vel*(2.5, {1.5*sqrt(3)})$)--($\vel*(3.5, {1.5*sqrt(3)})$);
		\fill[fill=gray!25] ($\vel*(2.5, {2.5*sqrt(3)})$)--($\vel*(2, {2*sqrt(3)})$)--($\vel*(3, {2*sqrt(3)})$);
		\foreach \k in {0, 1, ...,\rows} {
			\draw ($\vel*\k*(0.5, {0.5 * sqrt(3)})$) -- ($\vel*(\rows, 0) + \vel*\k*(-0.5, {0.5 * sqrt(3)})$);
			\draw ($\vel*\k*(1, 0)$) -- ($\vel*(\rows/2,{\rows/2 * sqrt(3)}) + \vel*\k*(0.5, {-0.5 * sqrt(3)})$);
			\draw ($\vel*\k*(1, 0)$) -- ($(0,0) + \vel *\k*(0.5, {0.5 * sqrt(3)})$);
		}
% ####################   prva vrsta    ####################
		\filldraw[red] (0, 0) circle (5pt);
		\filldraw[red] (\vel, 0) circle (5pt);
		\filldraw[green] (2*\vel, 0) circle (5pt);
		\filldraw[green] (3*\vel, 0) circle (5pt);
		\filldraw[red] (4*\vel, 0) circle (5pt);
		\filldraw[green] (5*\vel, 0) circle (5pt);
		\filldraw[green] (6*\vel, 0) circle (5pt);
% ####################   druga vrsta    ####################	
		\filldraw[blue] ($\vel*(0.5, {0.5*sqrt(3)})$) circle (5pt);
		\filldraw[red] ($\vel*(1.5, {0.5*sqrt(3)})$) circle (5pt);
		\filldraw[red] ($\vel*(2.5, {0.5*sqrt(3)})$) circle (5pt);
		\filldraw[green] ($\vel*(3.5, {0.5*sqrt(3)})$) circle (5pt);
		\filldraw[green] ($\vel*(4.5, {0.5*sqrt(3)})$) circle (5pt);
		\filldraw[blue] ($\vel*(5.5, {0.5*sqrt(3)})$) circle (5pt);
% ####################   tretja vrsta    ####################
		\filldraw[red] ($\vel*(1, {sqrt(3)})$) circle (5pt);
		\filldraw[green] ($\vel*(2, {sqrt(3)})$) circle (5pt);
		\filldraw[red] ($\vel*(3, {sqrt(3)})$) circle (5pt);
		\filldraw[blue] ($\vel*(4, {sqrt(3)})$) circle (5pt);
		\filldraw[green] ($\vel*(5, {sqrt(3)})$) circle (5pt);
% ####################   četrta vrsta    ####################
		\filldraw[red] ($\vel*(1.5, {1.5*sqrt(3)})$) circle (5pt);
		\filldraw[green] ($\vel*(2.5, {1.5*sqrt(3)})$) circle (5pt);
		\filldraw[blue] ($\vel*(3.5, {1.5*sqrt(3)})$) circle (5pt);
		\filldraw[blue] ($\vel*(4.5, {1.5*sqrt(3)})$) circle (5pt);
% ####################   peta vrsta    ####################
		\filldraw[red] ($\vel*(2, {2*sqrt(3)})$) circle (5pt);
		\filldraw[green] ($\vel*(3, {2*sqrt(3)})$) circle (5pt);
		\filldraw[green] ($\vel*(4, {2*sqrt(3)})$) circle (5pt);
% ####################   šesta vrsta    ####################
		\filldraw[blue] ($\vel*(2.5, {2.5*sqrt(3)})$) circle (5pt);
		\filldraw[blue] ($\vel*(3.5, {2.5*sqrt(3)})$) circle (5pt);
% ####################   sedma vrsta    ####################
		\filldraw[blue] ($\vel*(3, {3*sqrt(3)})$) circle (5pt);
% ####################   obrobe za lepši izgled    ####################
		\foreach \k in {0, 1, ...,\rows} {
			\foreach \j in {0, ...,\k} {
				\draw ($\vel*(\rows-\k, 0) + \vel*\j*(0.5, {0.5*sqrt(3)})$) circle (5pt);
			}
		}
% ####################   napis za pravilo barvanja    ####################	
		\draw ($\vel*(-0.8+\rows/4, {0.3+\rows/4 * sqrt(3)})$) node[rotate=60] {\text{Vozlišča pobarvana zgolj z rdečo in modro.}};
		\draw ($\vel*(0.8+3*\rows/4, {0.3+\rows/4 * sqrt(3)})$) node[rotate=-60] {Vozlšča pobarvana zgolj z modro in zeleno.};
		\draw (\vel*\rows/2, -1) node {Vozlišča pobarvana zgolj z rdečo in zeleno.};
	\end{tikzpicture}
	\caption{Vidimo Spernerjevo barvanje s tremi popolno pobarvanimi trikotniki, ki smo jih zaradi preglednosti osenčili. Zaradi lepše predstave, smo barve namesto s številčnimi vrednosti ponazorili s pravimi barvami \protect{\cite[str.\ 4, slika 1.4]{Schaefer2014}}.}
\end{figure}
Za $n$-simpleks, ki ima oglišča pobarvana z vsemi barvami iz množice $\{ 0, 1, \dots, n \}$, pravimo, da je \emph{popolno pobarvan}. Če simpleks pobarvamo s Spernerjevim barvanjem smo lahko prepričani, da je popolno pobarvan. Naslednji izrek izpostavi še eno bolj zanimivo lastnost takega barvanja.
\begin{lema}[Spernerjeva lema~\protect{\cite[str.\ 10, izrek 4]{Ahlbach}}]\label{izr:sperner}
Vsaka triangulacija $k$-simpleksa s Spernerjevim barvanjem vsebuje liho število popolno pobarvanih $k$-simpleksov.
\end{lema}
\begin{proof}
Naj bo $S$ $k$-simpleks in naj bo $T$ njegova triangulacija. Lemo bomo dokazovali z indukcijo na dimenzijo $k$ simpleksa $S$. \\
\underline{Baza indukcije ($k = 0$).}
Če je $k=0$, lema očitno drži, saj je $0$-simpleks točka. Pri Spernerjevem barvanju točke, imamo samo eno možnost in tako dobimo en popolno pobarvan $0$-simpleks.\\
\underline{Indukcijska predpostavka ($k - 1$).}
Predpostavimo, da lema drži v dimenziji $(k - 1)$. Torej v vsaki triangulaciji $(k - 1)$-simpleksa s Spernerjevim barvanjem najdemo liho število $(k - 1)$-simpleksov, ki so popolno pobarvani. \\
\underline{Indukcijski korak ($(k - 1) \Longrightarrow k$).}
Poskusimo dokazati pravilnost izjave tudi v dimenziji $k$. Vzemimo poljubno triangulacijo $T$ $k$-simpleksa $S$. Naj bodo vozlišča triangulacije $T$ pobarvana s Spernerjevim barvanjem. Označimo število vseh popolno pobarvanih $i$-simpleksov iz $T$ s $S_p^i$, število tistih popolno pobarvanih $i$-simpleksov, ki se nahajajo na robu simpleksa $S$ pa z $R_p^i$. Vsak popolno pobarvan $k$-simpleks iz triangulacije T vsebuje natanko en popolno pobarvan $(k-1)$-simpleks, medtem ko ostali simpleksi lahko vsebujejo dva ali nobenega. Če $k$-simpleks vsebuje vozlišča pobarvana z vsemi barvami $1, 2, \dots, k-1$, kjer se ena barva ponovi dvakrat, tak simpleks vsebuje dva popolno pobarvana simpleksa. Ostali simpleksi pa ne vsebujejo popolno pobarvanih $(k - 1)$-simpleksov. Iz zgornjega lahko sklepamo, da je $S_p^k \equiv S_p^{k-1} \pmod 2$. Pri štetju popolno pobarvanih $(k - 1)$-simpleksov smo vsakega iz notranjosti $S$ šteli dvakrat, tiste iz roba $S$ pa samo enkrat. Zato velja tudi $S_p^{k-1} \equiv R_p^{k - 1} \pmod 2$. Torej velja $S_p^k \equiv R_p^{k - 1} \pmod 2$. Popolno pobarvane $(k - 1)$-simplekse zaradi Spernerjevega barvanja lahko najdemo zgolj na enem pravem licu $L$ simpleksa $S$. Po indukcijski predpostavki vsebuje $L$ liho mnogo popolno pobarvanih $(k - 1)$-simpleksov, kar pomeni, da je $R_p^{k - 1}$ liho število. Torej je tudi  $S_p^k$ lih.
\end{proof}
V posebnem izrek~\ref{izr:sperner} pove, da vsaka triangulacija simpleksa $S$ s Spernerjevim barvanjem vsebuje vsaj en popolno pobarvan simpleks $R$. Ker sta simpleksa $S$ in $R$ oba popolno pobarvana, si lahko predstavljamo, da se nekatere lastnosti prenesejo iz $S$ na $R$. To dejstvo lahko uporabimo, pri dokazovanju nekaterih trditev in tudi za nas bo odigrala pomembno vlogo pri dokazu izreka~\ref{izr:PM}. 

Poleg simpleksov, bomo spoznali še ene enostavne množice. To so kocke. Tudi z njimi, si bomo lahko zaradi enostavnosti velikokrat pomagali pri dokazovanju topoloških rezultatov. 
\begin{definicija}
Naj bo $a \in \R$ poljubno pozitivno realno število in naj bo interval $\I = \left [ -a, a \right ] $. Potem je $C := \I^n$ \emph{$n$-dimenzionalna kocka} v $\R^n$ oziroma \emph{$n$-kocka}.
\end{definicija}
Lastnosti, ki smo jih obravnavali pri simpleksih bomo posplošili na kocke. Nekatere lastnosti izgledajo skoraj enako za simplekse in za kocke, a kot bomo videli, moramo biti pri posploševanju previdni, saj lahko pride do sprememb ravno tam, kjer jih ne pričakujemo.
Triangulacijo definiramo enako, kot v primeru simpleksov.
\begin{definicija}
\emph{Triangulacija $T$ kocke} $C$ je taka množica simpleksov, za katero veljata naslednji lastnosti:
\begin{enumerate}
\item Unija vseh simpleksov iz $T$ enaka $C$.
\item Neprazen presek $L$ dveh simpleksov je lice obeh simpleksov in velja $L \in T$.
\end{enumerate}
\end{definicija}
Tako kot pri simpleksih, tudi pri kockah lahko barvamo vozlišča triangulacije.
\begin{definicija}
Naj bo podana $n$-kocka $C \in \R^n$ s triangulacijo $T$. Označimo množico vseh vozlišč triangulacije $T$ z $V$. \emph{Barvanje triangulacije} $T$ je preslikava $b : V \to \left \{0, 1, \dots, n \right \}$.
\end{definicija}
Tudi pri kockah bi si želeli tako pravilo za barvanje triangulacije kocke, da bi vsaka triangulacija kocke vsebovala vsaj en popolno pobarvan simpleks. Ker si želimo popolno pobarvan simpleks, nam ni potrebno vsakega oglišča kocke pobarvati s svojo barvo, saj je za $n$-kocko dovolj že $n+1$ barv.  Pravilo za Spernerjevo barvanje simpleksa lahko posplošimo na $n$-kocko na naslednji način.
Definirajmo naivno pravilo:
\begin{enumerate}
\item Na ogliščih so uporabljene vse barve iz množice $\{0, 1, \dots, n \}$,
\item vsako vozlišče $v$ iz roba kocke $C$ je enako pobarvano kot eno izmed oglišč, ki določajo lice, na katerem leži vozlišče $v$.
\end{enumerate}
S primerom na sliki \ref{fig:ni-pop} pokažimo, da barvanje, ki zadošča naivnemu pravilu nima želene lastnosti.
%####################   Naivno barvanje kocke brez popolno pobarvanega simpleksa   ####################
\begin{figure}[h!]
	\centering
	\begin{tikzpicture}%[scale=0.9]
		\foreach \x in {0, 1.5, 3, 4.5}
		{\foreach \y in {0,1.5,3,4.5}
			{\draw (0, \y ) -- (4.5, \y );
			\draw (\x ,0) -- (\x ,4.5);
			\draw ( 0 , \x) -- ( 4.5 - \x, 4.5);
			\draw ( \x , 0) -- ( 4.5, 4.5 - \x );
			}
		}		
			
		\filldraw[blue] (0, 0) circle (5pt);
		\filldraw[blue] (1.5, 0) circle (5pt);
		\filldraw[red] (3, 0) circle (5pt);
		\filldraw[red] (4.5, 0) circle (5pt);
		\filldraw[red] (0, 1.5) circle (5pt);
		\filldraw[blue] (1.5, 1.5) circle (5pt);
		\filldraw[red] (3, 1.5) circle (5pt);
		\filldraw[green] (4.5, 1.5) circle (5pt);
		\filldraw[blue] (0, 3) circle (5pt);
		\filldraw[red] (1.5, 3) circle (5pt);
		\filldraw[red] (3, 3) circle (5pt);
		\filldraw[red] (4.5, 3) circle (5pt);
		\filldraw[red] (0, 4.5) circle (5pt);
		\filldraw[red] (1.5, 4.5) circle (5pt);
		\filldraw[green] (3, 4.5) circle (5pt);
		\filldraw[green] (4.5, 4.5) circle (5pt);
		\foreach \x in {0, 1.5, 3, 4.5}
		{\foreach \y in {0, 1.5, 3, 4.5}
			{\draw (\x,\y) circle (5pt);
			}
		}	
	\end{tikzpicture}
	\caption{Triangulacija $n$-simpleksa $S$ pobarvana po naivnem pravilu ne vsebuje popolno pobarvanega $n$-simpleksa. Zaradi lepše slike, smo barvo $0$ prikazali z modro, barvo $1$ z rdečo, $2$ pa z zeleno.}\label{fig:ni-pop}
\end{figure}
Naivno pravilo lahko dopolnimo, tako da bo vsaka triangulacija $n$-simpleksa pobarvana po tem pravilu vsebovala popolno pobarvan $n$-simpleks. Če pogledamo dokaz izreka \ref{izr:sperner}, vidimo, da je bilo ključno dejstvo, da je bilo na robu liho število popolno pobarvanih simpleksov. Na sliki~\label{fig:ni-pop} pa temu ni tako. Popolno pobarvani $1$-simpleksi so tisti, ki so pobarvani z modro in rdečo. Simpleks $S$ na robu vsebuje $4$ popolno pobarvane $1$-simplekse. Če naivnemu barvanju dodamo pogoj, ki nam za vsako triangulacijo $n$-kocke zagotavlja liho število $(n-1)$-simpleksov na robu dobimo Spernerjevo barvanje za kocke.
\begin{definicija}\label{def:cubsperner}
Spernerjevo barvanje triangulacije $n$-kocke $C$ z vozlišči $V$ je preslikava $b : V \to \left \{ 0, 1, \dots, n \right \}$, pri kateri so izpolnjeni trije pogoji.
\begin{enumerate}
\item Na ogliščih so uporabljene vse barve iz množice $\{ 0, 1, \dots, n \}$,
\item kocka $C$ vsebuje liho mnogo popolno pobarvanih pravih lic,
\item vsako vozlišče $v$ iz lica $L$ kocke $C$ je enako pobarvano kot eno izmed oglišč, ki ležijo na licu $L$.
\end{enumerate}
\end{definicija}
Definicija Spernerjevega barvanja je precej bolj komplicirana, kot definicija Spernerjevega barvanja za simplekse, saj je definicija \label{def:cubsperner} definirana rekurzivno.
Pokazali bomo, da vsaka triangulacija $n$-kocke pobarvana s Spernerjevim barvanjem vsebuje vsaj en popolno pobarvan $n$-simpleks.
\begin{lema}[Spernerjeva lema za kocke]\label{izr:kubsperner}
V vsakem Spernerjevem barvanju triangulacije $k$-kocke je liho mnogo popolno pobarvanih $k$-simpleksov.
\end{lema}
\begin{proof}
Naj bo $C$ $k$-kocka in naj bo $T$ njena triangulacija. Lemo bomo dokazovali z indukcijo na dimenzijo kocke $k$. \\
\underline{Baza indukcije ($k = 0$).}
Če je $k=0$, lema očitno drži, saj je $0$-kocka točka. Pri Spernerjevem barvanju točke, pa imamo samo eno možnost in tako dobimo eno popolno pobarvano $0$-kocko.\\
\underline{Indukcijska predpostavka ($k - 1$).}
Predpostavimo, da lema drži v dimenziji $(k - 1)$. Torej v vsaki triangulaciji $(k - 1)$-kocke s Spernerjevim barvanjem najdemo liho število $(k - 1)$-simpleksov, ki so popolno pobarvani. \\
\underline{Indukcijski korak ($(k - 1) \rightarrow k$).}
Pravilnost izjave želimo dokazati tudi v dimenziji $k$. Vzemimo poljubno triangulacijo $T$ $k$-kocke $C$. Naj bodo vozlišča triangulacije $T$ pobarvana s Spernerjevim barvanjem. Označimo število vseh popolno pobarvanih $i$-simpleksov iz $T$ s $C_p^i$, število tistih popolno pobarvanih $i$-simpleksov, ki se nahajajo na robu simpleksa $S$ pa z $R_p^i$. Vsak popolno pobarvan $k$-simpleks vsebuje natanko en popolno pobarvan $(k-1)$-simpleks, medtem ko ostali simpleksi lahko vsebujejo dva ali pa nobenega. Če $k$-simpleks vsebuje vozlišča pobarvana z vsemi barvami $0, 1, \dots, k-1$, kjer se ena barva ponovi dvakrat, tak simpleks vsebuje dva popolno pobarvana simpleksa. Ostali simpleksi pa zagotovo ne vsebujejo popolno pobarvanih $(k - 1)$-simpleksov, saj niti ne vsebujejo vseh barv iz množice $\{0, 1, \dots, k-1 \}$. Iz zgornjega lahko sklepamo, da je
\begin{equation}\label{eq:kongr1}
S_p^k \equiv S_p^{k-1} \pmod 2.
\end{equation}
Pri štetju popolno pobarvanih $(k - 1)$-simpleksov smo vsakega iz notranjosti $C$ šteli dvakrat, tiste iz roba $C$ pa samo enkrat. Torej velja kongruenca 
\begin{equation}\label{eq:kongr2}
S_p^{k-1} \equiv R_p^{k - 1} \pmod 2.
\end{equation}
Iz enačbe~\ref{eq:kongr1} in enačbe~\ref{eq:kongr2} sledi, da je $S_p^k \equiv R_p^{k - 1} \pmod 2$.
Ker je kocka pobarvana s Spernerjevim barvanjem ima liho število popolno pobarvanih lic, na njih pa zaradi indukcijske predpostavke najdemo liho število popolno pobarvanih $(k - 1)$-simpleksov. Ker na drugih licih ni popolno pobarvanih $(k - 1)$-simpleksov je število $R_p^{k - 1}$ liho. Zato je tudi $S_p^k$ lih.
\end{proof}
Enako kot pri simpleksih, to pomeni, da se nekatere lastnosti prenašajo iz celotne kocke $C$ na manjše simplekse, kar bo za nas zelo pomembno.

%
%####################     3. POGLAVJE: POINCARÉ-MIRANDOV IZREK     ####################
\section{Poincar\'e-Mirandov izrek}\label{raz:PM}
Sedaj se bomo posvetili ničlam preslikav, tj. takim točkam, ki jih neka preslikava $f : \I^n \to \R^n$ slika v nič. Iskali bomo torej odgovor na vprašanje: \citat{Kakšni pogoji nam zagotavljajo obstoj take točke $c$, za katero je $f(c) = (0, 0, \dots, 0)$?}. Zaradi krajšega zapisa bomo točko $(0, 0, \cdots, 0) \in \R^n$, ko ne bo dvomov o dimenziji, označevali z oznako $\0$. V primeru funkcije $f : \I \to \R$ nam odgovor na vprašanje ponudi Bolzanov izrek o srednji vrednosti.
\begin{izrek}[Bolzanov izrek o srednji vrednosti]\label{izr:bolzano}
Izberimo poljubno pozitivno realno število $a$ in definirajmo interval $\I := [-a, a]$. Za zvezno funkcijo $f : \I \to \R$, za katero velja $f(-a) \leq 0 \leq f(a)$, obstaja točka $c \in [-a, a]$, da je $f(c) = 0$.
\end{izrek}
Intuitivno nam izrek pove, da ne moremo povezati točke iz spodnje polravnine s točko iz zgornje polravnine, ne da bi pri tem sekali $x$-os, kar se lepo vidi iz slike~\ref{fig:bolzano}. Izrek nam ne pove, kako lahko iskano točko najdemo, nam pa samo iz vrednosti funkcije na robu pove kdaj lahko z gotovostjo trdimo, da ima funkcija ničlo. Ključno vlogo, pa ima seveda zveznost funkcije.
\begin{dokaz}[Dokaz izreka~\ref{izr:bolzano}]
Če je $f(-a) = 0$ ali pa je $f(a) = 0$, smo tako točko že našli. V nasprotnem predpostavimo, da ne obstaja tako število $c \in \I$, za katerega je vrednost funkcije $f(c) = 0$. Definiramo množici $A := f^{-1}(- \infty, 0)$ in $B := f^{-1}(0, \infty)$. Množici sta disjunktni, saj sta prasliki disjunktnih množic. Zaradi zveznosti funkcije $f$ sta množici tudi odprti. Hitro lahko vidimo, da sta neprazni, saj je $-a \in A$ in $a \in B$. Unija množic $A$ in $B$ je enaka intervalu $\I$, zato množici $A$ in $B$ razdelita interval $\I$ na dve disjunktni množici, kar pa je protislovje, saj vemo, da je interval povezana množica. Torej je bila naša predpostavka, da ne obstaja tak $c$, da je $f(c) = 0$ napačna.
\end{dokaz}

\begin{figure}[h!]
	\centering
	\begin{tikzpicture}
% ####################   KOORDINATNI SISTEM    ####################
		\draw[->] (-4, 0) -- (4, 0);
		\draw[->] (0, -2.5) -- (0, 3);
		\draw (-2, -0.1) -- (-2, 0.1);
		\draw (2, -0.1) -- (2, 0.1);
		\node at (3.8, -0.3) {$x$};
		\node at (-0.3, 2.8) {$y$};
		\node at (-2, 0.3) {$-a$};
		\node at (2, -0.3) {$a$};
% ####################   FUNKCIJA    ####################
		\draw[scale=1,domain=-2:2,smooth,variable=\x,black] plot ({\x},{0.2 * pow(\x, 4) + 0.2 * pow(\x, 3) - 0.6 * pow(\x, 2) -0.2});
		\draw[dotted] (-2, 0) -- (-2, -1);
		\draw[dotted] (2, 0) -- (2, 2.2);
		\filldraw[black] (-2, -1) circle (2pt) node[black, left]{$f(-a)$};
		\filldraw[black] (2, 2.2) circle (2pt) node[black, right]{$f(a)$};
		\filldraw[red] (1.43, 0) circle (2pt) node[black, above left]{$c$};
	\end{tikzpicture}
	\caption{Slika prikazuje dogajanje, ki ga opisuje izrek~\ref{izr:bolzano}}\label{fig:bolzano}
\end{figure}

Zgornji rezultat bi si želeli posplošiti na višje dimenzije. Večdimenzionalni analog funkcijam so preslikave, intervalom pa $n$-kocke. Ostane le še vprašanje, kako bi posplošili pogoje, da bi imela preslikava podobno lastnost kot zgoraj. Razmislimo, kako bi posplošili izrek na zvezno preslikavo $f : \I^n \to \R^n$. Najprej lahko preslikavo zapišemo s pomočjo komponentnih funkcij $f = (f_1, f_2, \dots, f_n)$. Da lažje ugotovimo, kakšne pogoje bi izbrali, najprej obravnavamo malo poenostavljen primer. Kakšne pogoje bi postavili, če bi bila vsaka koordinatna funkcija $f_i$ odvisna samo od ene koordinate točke $x$, npr.\ $x_i$? Potem lahko vsako koordinatno funkcijo $f_i$ zapišemo kot $f_i(x_1, x_2, \dots, x_n) = g_i(x_i)$, kjer so funkcije $g_i : \I \to \R$. V tem primeru za vsako funkcijo $g_i$ uporabimo enake zahteve, kot v izreku \ref{izr:bolzano}, kar pa nam za funkcije $f_i$ porodi naslednji pogoj: 
\begin{equation}\label{pog:ndimbolzano}
f_i(x_1, \dots, x_{i-1}, -a, x_{i+1}, \dots, x_n) < 0 < f_i(x_1, \dots, x_{i-1}, a, x_{i+1}, \dots, x_n).
\end{equation}
Izrek~\ref{izr:bolzano} nam za vsako funkcijo $g_i : \I \to \R$ zagotavlja obstoj take točke $c_i \in \I$, da je $g_i(c_i) = 0$. To pa pomeni, da je
$$f(c_1, \dots, c_n) = (f_1(c_1, \dots, c_n), \dots, f_n(c_1, \dots, c_n)) = (g_1(c_1), \dots, g_n(c_n)) = \0.$$
Izrek \ref{izr:PM} pokaže, da pogoj~\eqref{pog:ndimbolzano} zadošča tudi v primeru poljubno zapletene zvezne preslikave $f : \I^n \to \R^n$. Preden pa navedemo izrek zapišimo pogoj~\eqref{pog:ndimbolzano} na način, ki bolj spodbudi geometrijsko predstavo, saj stvari lažje razumemo in si jih zapomnimo, če si jih znamo čim bolj slikovito predstavljati.
Naj bo $a$ poljubno pozitivno realno število in naj bo $\I^n := [-a, a]^n$ kocka v $\R^n$. Stranske ploskve ali lica kocke $\I^n$ označimo z $\I_i^- = \{x\in \I^n | x_i = -a\}$ in $\I_i^+ = \{x\in \I^n | x_i = a\}.$ Za preslikavo $f = (f_1, f_2, \dots, f_n) : \I^n \to \R^n$, lahko pogoj~\eqref{pog:ndimbolzano} zapišemo na naslednji način:
\begin{equation}\label{pog:PM}
f_i(\I_i^-) \subset (- \infty, 0]  \text{ in } f_i(\I_i^+) \subset ]0, \infty) \text{, za vsak } i \in  \{1, \dots, n\}.
\end{equation}

%Preden navedemo izrek, ponovimo eno pomembno lastnost množic. Podobno, kot smo simplekse razdelili na manjše simplekse, lahko poljubno podmnožico evklidskega prostora $\R^n$ zapišemo kot unijo množic.
%\begin{definicija}
%Družini $\mathcal{A}$ podmnožic množice $X$, za katero velja $\bigcup_{A \in \mathcal{A}} = X$ pravimo \emph{pokritje množice} $X$. Če družina $\mathcal{A}$ vsebuje zgolj odprte množice, jo imenujemo \emph{odprto pokritje množice} $X$. Pokritje $\mathcal{B}$ množice $X$, ki vsebuje samo množice iz $\mathcal{A}$, je \emph{podpokritje} pokritja $\mathcal{A}$.
%\end{definicija}

%\begin{definicija}
%Množica $K \in \R^n$ je kompaktna, če za vsako odprto pokritje $\mathcal{A}$ množice $K$ obstaja končno podpokritje $\mathcal{B}$ množice $K$.
%\end{definicija}


\begin{izrek}[Poincar\'e-Mirandov izrek \protect{\cite{Kulpa}}]\label{izr:PM}
Naj bo $a$ poljubno pozitivno realno število in naj bo $\I^n := [-a, a]^n$ kocka v $\R^n$. Naj bo $f = (f_1, f_2, \dots, f_n) : \I^n  \to \R^n$ taka zvezna preslikava iz kocke v evklidski prostor $\R^n$, ki ustreza pogoju~\eqref{pog:PM}. Potem obstaja točka $c \in \I^n$, da je vrednost preslikave $f(c) = \0$.
\end{izrek}

\begin{figure}[h!]
	\centering
	\begin{tikzpicture}
% ####################   1 - D    ####################
		\filldraw[black] (0, 0) circle (2pt) node[black, left]{$\I_1^-$};		
		\filldraw[black] (2, 0) circle (2pt) node[black, right]{$\I_1^+$};
		\draw (0, 0) -- (2, 0);
% ####################   2 - D    ####################
		\draw (4, -1) -- (6, -1);
		\draw (4, -1) -- (4, 1);
		\draw (6, -1) -- (6, 1);
		\draw (4, 1) -- (6, 1);
		\draw (4, 0) node[black, left]{$\I_1^-$};
		\draw (6, 0) node[black, right]{$\I_1^+$};
		\draw (5, -1) node[black, below]{$\I_2^-$};
		\draw (5, 1) node[black, above]{$\I_2^+$};
% ####################   3 - D    ####################
		\draw (8, -1) -- (10, -1);
		\draw (10, -1) -- (10, 1);
		\draw (10, 1) -- (8, 1);
		\draw (8, 1) -- (8, -1);
		\draw[dashed] (8.7, -0.3) -- (10.7, -0.3);
		\draw (10.7, -0.3) -- (10.7, 1.7);
		\draw (10.7, 1.7) -- (8.7, 1.7);
		\draw[dashed] (8.7, 1.7) -- (8.7, -0.3);
		\draw[dashed] (8, -1) -- (8.7, -0.3);
		\draw (10.7, -0.3) -- (10, -1);
		\draw (10, 1) -- (10.7, 1.7);
		\draw (8, 1) -- (8.7, 1.7);
		\draw (8, 0.3) node[gray!120, right]{$\I_1^-$};
		\draw (10, 0.3) node[black, right]{{\large $\I_1^+$\par}};
		\draw (9.15, -0.35) node[black, above]{{\large $\I_2^-$\par}};
		\draw (9.65, 0.25) node[gray!120, above]{$\I_2^+$};
		\draw (9.4, -1.05) node[gray!120, above]{$\I_3^-$};
		\draw (9.4, 0.9) node[black, above]{{\large $\I_3^+$\par}};
	\end{tikzpicture}
	\caption{Na sliki, lahko vidimo, kako smo ozačili posamezno lice kocke. Na levi je kocka $\I$, na sredini kocka $\I^2$, na desni pa je prikazana kocka $\I^3$ \protect{\cite[slika 1.2]{Ahlbach}}.}
\end{figure}

\begin{dokaz}
Imamo zvezno preslikavo $f : \I^n \to  \R^n$, ki zadošča pogoju~\eqref{pog:PM}. Za vsako število $i \in \{ 1, \dots, n\}$ definiramo množici $H_i\ ^- := f_i^{-1} (-\infty, 0]$ in $H_i\ ^+ := f_i^{-1} [0, -\infty)$. Najprej bomo pokazali, da je za dokaz izreka dovolj, če za vsako naravno število $k \in \N$ poiščemo $n$-simpleks $S_k$ iz $k$-baricentrične triangulacije $T_k$ z lastnostjo 
\begin{equation}\label{eq:nonempty}
S_k \cap H_i^- \neq \emptyset \neq S_k \cap H_i^+, \text{ za vsak } i \in \{ 1, \dots, n \}.
\end{equation}
Kasneje bomo pokazali, da taki simpleksi res obstajajo.

Pokažimo torej, da bi obstoj takih simpleksov res zaključil dokaz. Recimo, da imamo za vsak $k \in \N$ $n$-simpleks $S_k \in T_k$ z lastnostjo~\eqref{eq:nonempty}. V vsakem simpleksu $S_k$ si lahko izberemo točko $x_k$ in tako dobimo neskončno zaporedje $\left ( x_k \right ) _{k = 1}^{\infty}$. Ker zaporedje leži v kompaktnem prostoru $\I^n$, obstaja konvergentno podzaporedje $\left ( x_{k_j} \right ) _{j = 1}^{\infty}$ z limito $c \in \I^n$. Izberimo si neki $i \in \left \{1, 2, \dots, n\right \}$. Zaradi lastnosti ~\eqref{eq:nonempty} lahko za vsak $j \in \N$ poiščemo točki $z_j \in S_{k_j} \cap H_i^-$ in $w_j \in S_{k_j} \cap H_i^+$. Ker velja $\lim\limits_{j \to \infty} \diam(S_{k_j}) = 0$ je limita  $\lim\limits_{j \to \infty} z_j = \lim\limits_{j \to \infty} w_j = \lim\limits_{j \to \infty} x_j = c$ (slika~\ref{fig:istalimita}). Funkcija $f$ zvezna, zato je tudi zaporedje $f_i(z_j)$ konvergentno z limito $f_i(c)$. Členi tega zaporedja so vsi manjši ali enaki nič, saj leži zaporedje v množici $H_i^-$, zato je tudi $f_i(c) \leq 0$. Podobno lahko sklepamo, da je zaporedje $f_i(w_j)$ konvergentno z limito $f_i(c)$. Ker so vsi členi $f_i(w_j)$ nenegativni, je tudi limita $f_i(c) \geq 0$. Ugotovili smo, da je $0 \leq f_i(c) \leq 0$, torej je $f_i(c) = 0$. Razmislek velja za vsak $i \in \left \{1, 2, \dots, n\right \}$, zato velja $f(c) = \0$.

\begin{figure}[h!]
	\centering
	\begin{tikzpicture}
		\draw (-3, -3) -- (3, -3);
		\draw (3, -3) -- (3, 3);
		\draw (3, 3) -- (-3, 3);
		\draw (-3, 3) -- (-3, -3);	
		
		\draw (-2.2, -2.7) -- (-0.2, -2.7);
		\draw (-0.2, -2.7) -- (-2.2, -0.7);
		\draw (-2.2, -2.7) -- (-2.2, -0.7);
		\filldraw[blue] (-0.6, -2.45) circle (1.5pt);	
		\filldraw[red] (-2.05, -1) circle (1.5pt);
		
		\draw (0.5, -2.2) -- (2, -2.2);
		\draw (2, -2.2) -- (0.5, -0.7);
		\draw (0.5, -0.7) -- (0.5, -2.2);			
		\filldraw[blue] (0.7, -2) circle (1.5pt);	
		\filldraw[red] (1.6, -2) circle (1.5pt);
		
		\draw (1.2, -0.6) -- (2.2, -0.6);
		\draw (2.2, -0.6) -- (1.2, 0.4);
		\draw (1.2, 0.4) -- (1.2, -0.6);
		\filldraw[blue] (1.35, -0.4) circle (1.5pt);	
		\filldraw[red] (1.6, -0.15) circle (1.5pt);
		
		\draw (1.5, 0.5) -- (1.5, 1.1);
		\draw (1.5, 1.1) -- (0.9, 1.1);
		\draw (0.9, 1.1) -- (1.5, 0.5);	
		\filldraw[red] (1.15, 1) circle (1.4pt);	
		\filldraw[blue] (1.35, 0.8) circle (1.4pt);
		
		\draw (0.7, 0.8) -- (0.7, 1.2);
		\draw (0.7, 1.2) -- (0.3, 1.2);
		\draw (0.3, 1.2) -- (0.7,0.8);
		\filldraw[blue] (0.455, 1.14) circle (1.1pt);	
		\filldraw[red] (0.64, 1) circle (1.1pt);
		
		\draw (0.1, 1) -- (0.4, 1);
		\draw (0.4, 1) -- (0.4, 0.7);
		\draw (0.4, 0.7) -- (0.1, 1);
		\filldraw[red] (0.25, 0.95) circle (0.8pt);
		\filldraw[blue] (0.35, 0.85) circle (0.8pt);
		
		
		\draw (-0.05, 0.8) -- (0.15, 0.8);
		\draw (0.15, 0.8) -- (0.15, 0.6);
		\draw (0.15, 0.6) -- (-0.05, 0.8);
		\filldraw[red] (0.055, 0.755) circle (0.5pt);
		\filldraw[blue] (0.105, 0.7) circle (0.5pt);	
		
		\draw (0, 0.65) -- (0, 0.55);
		\draw (0, 0.55) -- (0.1, 0.55);
		\draw (0.1, 0.55) -- (0, 0.65);
		\filldraw[red] (0.025, 0.6) circle (0.3pt);
		\filldraw[blue] (0.05, 0.57) circle (0.3pt);	
		
		\draw (-0.05, 0.55) -- (-0.05, 0.5);
		\draw (-0.05, 0.5) -- (0, 0.5);
		\draw (0, 0.5) -- (-0.05, 0.55);	
	\end{tikzpicture}
	\caption{Na sliki modre točke predstavljajo točke iz $H_i^+$, rdeče pa točke iz $H_i^-$. Ko se simpleksi zmanjšujejo, sta točki vedno bližje in zato imata zaporedji teh točk isto limito.}\label{fig:istalimita}
\end{figure}

Dokazati moramo še, da taki simpleksi $S_k$ res obstajajo. Tudi to bomo dokazovali v dveh delih. Najprej bomo definirali barvanje triangulacije kocke, nato pa bomo v prvem delu pokazali, da imajo popolno pobarvani simpleksi lastnost~\eqref{eq:nonempty}. Kasneje bomo pokazali, da je izbrano barvanje Spernerjevo, kar nam v vsaki triangulaciji zagotavlja obstoj vsaj enega popolno pobarvanega simpleksa. 
Omenjeno barvanje $\varphi$ triangulacije $n$-kocke $C$ z množico vozlišč $V$ definiramo na naslednji način:
$$\varphi : V \to \left \{ 0, 1, \dots, n \right \}$$
$$\varphi(x) = \max \left \{ j : x \in \bigcap_{i=0}^j F_i^+\right \}$$
Kjer je $ F_0^+ = \I^n$, $ F_i^+ =  H_i^+ \setminus \I_i^-$.

Trdimo, da ima pri tem barvanju popolno pobarvan simpleks $S$ lastnost~\eqref{eq:nonempty}. 
Za vsak $i \in \{1, 2 \dots, n \}$ in za vsaki taki točki $x$ in $y$ iz $\I^n$, za kateri je $\varphi(x) = i - 1$ in $\varphi(y) = i$, lahko sklepamo, da je $x \in H_{i}^-$ in $y \in H_i^+$. Res, denimo, da  $x \notin H_{i}^-$. Potem  $x \in F_i^+ = H_{i}^+ \setminus \I_i^-$, kar pa pomeni, da je $\varphi(x) = i$. To je protislovje, torej res velja $x \in H_{i}^-$. Podobno lahko sklepamo, če $y \notin H_i^+$, potem tudi $y \notin F_i^+$, kar pomeni, da je $\varphi(y) < i$. Spet protislovje s tem, da je $\varphi(y) = i$, torej je tudi $y \in H_i^+$.
Ker ima popolno pobarvan simpleks vsako vozlišče pobarvano s svojo barvo, so na ogliščih zastopane vse barve iz množice $\{ 0, 1, \dots, n \}$, zato ima tak simpleks res lastnost~\eqref{eq:nonempty}. 

Dokaz bo zaključen, če pokažemo, da je barvanje $\varphi$ Spernerjevo barvanje, saj bomo potem lahko v vsaki triangulaciji našli popolno pobarvan simpleks.
Pokažimo najprej, da je $n$-kocka $\I^n$ pri barvanju z barvanjem $\varphi$ pobarvana popolno. Pokazati moramo, da smo pri barvanju njenih oglišč uporabili vse barve iz množice $\{ 0, 1, \dots, n \}$ in da vsebuje liho število popolno pobarvanih pravih lic. Za boljšo predstavo dokaza, si lahko na sliki \ref{fig:PMcolor} pogledamo, kako izgleda barvanje oglišč v dimenzijah $0$, $1$, $2$ in $3$.
\begin{figure}[h!]
	\centering
	\begin{tikzpicture}
% ####################   0 - D    ####################
		\filldraw[black] (-2, 0) circle (2pt) node[black, above]{$0$};
% ####################   1 - D    ####################
		\filldraw[black] (0, 0) circle (2pt) node[black, above]{$0$};		
		\filldraw[black] (2, 0) circle (2pt) node[black, above]{$1$};
		\draw (0, 0) -- (2, 0);
% ####################   2 - D    ####################
		\draw (4, -1) -- (6, -1);
		\draw (4, -1) -- (4, 1);
		\draw (6, -1) -- (6, 1);
		\draw (4, 1) -- (6, 1);
		\filldraw[black] (4, -1) circle (2pt) node[black, below left]{$0$};		
		\filldraw[black] (6, -1) circle (2pt) node[black, below right]{$1$};
		\filldraw[black] (6, 1) circle (2pt) node[black, above right]{$2$};		
		\filldraw[black] (4, 1) circle (2pt) node[black, above left]{$0$};
% ####################   3 - D    ####################
		\draw (8, -1) -- (10, -1);
		\draw (10, -1) -- (10, 1);
		\draw (10, 1) -- (8, 1);
		\draw (8, 1) -- (8, -1);
		\draw[dashed] (8.7, -0.3) -- (10.7, -0.3);
		\draw (10.7, -0.3) -- (10.7, 1.7);
		\draw (10.7, 1.7) -- (8.7, 1.7);
		\draw[dashed] (8.7, 1.7) -- (8.7, -0.3);
		\draw[dashed] (8, -1) -- (8.7, -0.3);
		\draw (10.7, -0.3) -- (10, -1);
		\draw (10, 1) -- (10.7, 1.7);
		\draw (8, 1) -- (8.7, 1.7);
		\filldraw[black] (8, -1) circle (2pt) node[black, above left]{$0$};		
		\filldraw[black] (10, -1) circle (2pt) node[black, above left]{$1$};
		\filldraw[black] (10.7, -0.3) circle (2pt) node[black, above left]{$2$};		
		\filldraw[black] (8.7, -0.3) circle (2pt) node[black, above left]{$0$};
		\filldraw[black] (8, 1) circle (2pt) node[black, above left]{$0$};		
		\filldraw[black] (10, 1) circle (2pt) node[black, above left]{$1$};
		\filldraw[black] (8.7, 1.7) circle (2pt) node[black, above left]{$0$};		
		\filldraw[black] (10.7, 1.7) circle (2pt) node[black, above left]{$3$};
	\end{tikzpicture}
	\caption{Skrajno levo imamo eno točko ($0$-kocko), ki je z barvanjem $\varphi$ pobarvana z $0$. Malo bolj proti desni lahko vidimo, kako barvanje $\varphi$ označi oglišča $1$-kocke, naslednja je z $\varphi$ pobarvana $2$-kocka, skrajno desno pa je pobarvana $3$-kocka.}\label{fig:PMcolor}
\end{figure}
%
Dokazovali bomo z indukcijo na $n$.\\
\underline{Baza indukcije ($n = 0$).}
Če je $n=0$, imamo eno točko označeno z $0$, kar je popolno pobarvana $0$-kocka.\\
\underline{Indukcijska predpostavka ($n - 1$).}
Predpostavimo, da je $(n-1)$-kocka popolno pobarvana. \\
\underline{Indukcijski korak ($(n - 1) \Longrightarrow n$).}
Poskusimo dokazati pravilnost izjave tudi v dimenziji $n$. Lice $I_n^-$ je enako pobarvano, kot $(n-1)$-kocka, saj za njuni oglišči veljata enake vsebovanosti v množicah $F_i^+$. Po indukcijski predpostavki je lice $I_n^-$ popolno pobarvano, zato vsebuje vse barve iz množice $\{ 0, 1, \dots, n-1\}$. Oglišče $(a, a, \dots, a) \in \R^n$ je označeno z barvo $n$, saj je vsebovano v vseh množicah $F_i^+$ za $i = 1, 2, \dots, n$. To pomeni, da je $n$-kocka označena z vsemi oznakami iz množice $\{ 0, 1, \dots, n \}$. Prepričati se moramo le še, da ima liho število popolno pobarvanih pravih lic. 
Trdimo, da je lice $\I_n^-$ edino popolno pobarvano live kocke $\I^n$. Poglejmo najprej, če je lahko katero od lic $\I_i^+$ popolno pobarvano. Za vsak $x \in \I_i^+$ velja, da je $\varphi (x) \neq i-1$, zato lice $\I_i^+$ ne more vsebovati oglišč z vsemi barvami $\{ 0, 1, \dots, n - 1 \}$. Poskusimo poiskati popolno pobarvano lice med lici $\I_i^-$. Za vsako točko $x \in \I_i^-$ velja $\varphi (x) < i$. Če želimo, da lice $\I_i^-$ vsebuje vse barve iz množice $\{ 0, 1, \dots, n - 1 \}$, mora biti $i = n$. Torej je lice $\I_n^-$ res edino popolno pobarvano pravo lice kocke $\I^n$.

Prepričati se moramo samo še, da so vsa vozlišča iz nekega lica kocke $\I^n$ pobarvana enako, kot neko oglišče, ki to lice določa. Vemo, da lahko vsako lice $L$ dobimo kot presek pravih lic. Zaradi lažjega zapisa definiramo znak $\star$, za katerega je $\I_i^{\star} = \I^n$ za vsak $i = 1, 2, \dots, n$. S pomočjo nove oznake lahko vsako lice $L$ kocke $\I^n$ zapišemo kot 
$$L = \I_1^{\varepsilon_1} \cap \I_2^{\varepsilon_2} \cap \cdots \cap \I_n^{\varepsilon_n},$$
kjer je $\varepsilon_i \in \{ \star, -, + \}$ za vsak $i = 1, 2, \dots, n$. Naj bo $x \in L$ in naj bo $\varphi (x) = l$ za neki $l \in \{ 0, 1, \dots, n \}$. Vemo, da je $\varepsilon_j \neq -$ za $j \leq l$ in $\varepsilon_{l+1} \neq +$. Poglejmo si oglišče 
$$y = \I_1^{\rho_1} \cap \I_2^{\rho_2} \cap \cdots \cap \I_n^{\rho_n},$$
kjer je
\[  \rho_i =  \left\{
\begin{array}{cl}\vspace{3pt}
	\varepsilon_i &, \varepsilon_i \in \{ +, - \},\\
	+ &, (\varepsilon_i = \star) \land (i < l), \\
	- &, (\varepsilon_i = \star) \land (i > l). \\
\end{array} 
\right. \]
Lahko se prepričamo, da je $y \in L$ in $\varphi (y) = l$, kar pomeni, da smo našli oglišče na licu $L$, ki je enako pobarvano, kot točka $x$. Ker smo to storili za poljubno točko, je vsaka točka iz roba kocke $\I^n$ pobarvana enako kot eno od oglišč, ki določajo lice, na katerem leži točka $x$. Torej je $\varphi$ res Spernerjevo barvanje.
\end{dokaz}
Leta 1940 je K. Miranda dokazal, da je izrek~\ref{izr:PM} ekvivalenten znanemu Brouwerjevemu izreku o negibni točki, ki pravi naslednje.
\begin{izrek}[Brouwerjev izrek o negibni točki \protect{\cite{Kulpa}}]\label{izr:fixedpoint}
Denimo, da imamo dano kocko $C = [-1, 1]^n$ in zvezno preslikavo $f : C \to C$, potem obstaja točka $x \in C$, da je $f(x) = x$.
\end{izrek}
Izkaže se, da si pri dokazovanju nekaterih matematičnih trditev lažje pomagamo z izrekom~\ref{izr:PM}, kot z izrekom~\ref{izr:fixedpoint}. Poglejmo si, kako s pomočjo izreka~\ref{izr:PM} dokažemo izrek o negibni točki. 

\begin{dokaz}[Dokaz izreka \ref{izr:fixedpoint}]
Naj bo $C = [-1, 1]^n$ kocka v evklidskem prostoru $~R^n$ in naj bo $f : C \to C$ poljubna zvezna preslikava iz kocke nazaj vase. Definiramo funkcijo $g : C \to \R^n$ s predpisom $g(x) = x - f(x)$. Funkcija $g$ je zvezna in zadošča pogojem izreka~\ref{izr:PM}. Res, za vsak $x \in \I_i^-$ velja, da je $x_i = -1$ in $f_i(x) \geq -1$, zato je 
$$g_i(x) = x_i - f_i(x) \leq -1 - (-1) =0.$$
 Podobno lahko razmislimo, da je za vsak $x \in \I_i^+$ vrednost funkcije $g_i \geq 0$. Zato obstaja taka točka $z \in C$, da je $g(x) = 0$, kar pomeni, da je $f(x) = x$.
\end{dokaz}
Enostavnost dokaza izreka~\ref{izr:fixedpoint} pokaže, kako močno orodje za dokazovanje topoloških trditev je izrek~\ref{izr:PM}.


%####################     4. POGLAVJE: RAZŠIRITEV FUNKCIJE     ####################
\section{Razširitev funkcije}\label{raz:siritev}
Včasih se nam zgodi, da imamo podano funkcijo samo na nekem majhnem območju, želeli pa bi si, da bi bila domena funkcije večja. Poglejmo si poljubno množico $A \in \R^n$ in neko funkcijo $f : A \to \R$. Denimo, da je $U$ taka množica, da je $A \subset U$. Funkciji $F : U \to \R$, za katero je $F(x) = f(x)$ za vsak $x \in A$, pravimo \emph{razširitev funkcije} $f$ na množico $U$. Če funkcijo razširimo brez omejitev, se informacija o funkciji na prvotni množici popolnoma izgubi. Da to informacijo ohranimo, bomo gledali zgolj zvezne funkcije $f$ in zvezne razširitve $F$. Pri tem se ponudi vprašanje, ali lahko vsako zvezno funkcijo razširimo. S primerom pokažemo, da temu ni tako.

\begin{primer}
Funkcijo $f : (-2, 0) \cup (0, 2) \to \R$ definiramo s predpisom
\[  f(x) = \left \{
\begin{array}{ll}
	-1 &, x \in (-2, 0)\\
	1 &, x \in (0, 2). \\
\end{array} 
\right. \]

\begin{figure}[h!]
	\centering
	\begin{tikzpicture}
% ####################   KOORDINATNI SISTEM    ####################
		\draw[- latex] (-4, 0) -- (4, 0);
		\draw[-latex] (0, -2) -- (0, 2.5);
		\draw (-2, -0.1) -- (-2, 0.1);
		\draw (2, -0.1) -- (2, 0.1);
		\node at (3.8, -0.3) {$x$};
		\node at (-0.3, 2.8) {$y$};
		\node at (-2, 0.3) {$-2$};
		\node at (2, -0.3) {$2$};
		\node at (0.5, -1) {$-1$};
		\node at (-0.3, 1) {$1$};
% ####################   FUNKCIJA    ####################
		\draw[<->] (0, -1) -- (-2, -1);
		\draw[<->] (0, 1) -- (2, 1);
		\draw[dotted] (-2, 0) -- (-2, -1);
		\draw[dotted] (2, 0) -- (2, 1);
	\end{tikzpicture}
	\caption{Prikazan je graf zvezne funkcije, ki je ne moremo zvezno razširiti na $\R$.}
\end{figure}

Funkcija $f$ je zvezna na $(-2, 0) \cup (0, 2)$, ne moremo pa je zvezno razširiti na $\R$, saj razširitev v točki $x = 0$ ne bi bila zvezna.
\end{primer}

Izkaže se, da je za obstoj razširitve dovolj, če je zvezna funkcija definirana na kompaktni množici.

\begin{lema}[Razširitev zvezne funkcije]\label{lem:razsiritev}
Naj bo $A$ kompaktna podmnožica evklidskega prostora $\R^n$ in naj bo dana zvezna funkcija $f : A \to \R$. Potem obstaja zvezna funkcija $F : \R^n \to \R$, da za vsak $x \in A$, velja $F(x) = f(x)$.
\end{lema}

\begin{dokaz}
Včasih je pri dokazu obstoja neke stvari najlažje, če to stvar poiščemo in jo vsem pokažemo. Tako bomo tudi mi napisali predpis funkcije $F$, ki zvezno razširi funkcijo $f$. Seveda bi se lahko pri dokazu oprli na Tietzejev razširitveni izrek, a je to delo zasnovano tako, da se poskuša izogniti uporabi abstraktnejših topoloških izrekov. Denimo torej, da imamo kompaktno množico $A$ in zvezno funkcijo \mbox{$f : A \to \R$}. Brez izgube splošnosti lahko predpostavimo, da je funkcija $f$ pozitivna. Res, vemo, da je zvezna funkcija na kompaktni množici omejena, torej lahko prištejemo dovolj veliko število $C$, da je funkcija $f + C$ pozitivna. Zveznost funkcije $f$ pa je ekvivalentna zveznosti funkcije $f + C$.
Razširitveno funkcijo \mbox{$F : \R^n \to \R$} lahko definiramo s predpisom \protect{\cite[str.\ 257, vaja 4.1.F]{Engelking1989}}:
\[  F(x) = \left \{
\begin{array}{ll}
	\inf \left \{ f(a) + \frac{d(x, a)}{d(x, A)} - 1; a \in A \right \} &, x \in A^c \\
	f(x) &, x \in A. \\
\end{array} 
\right. \]
Enostavno se je prepričati, da je za vsak $x \in A$, $F(x) = f(x)$. Ugotoviti moramo le še, ali je funkcija $F$ res zvezna na $R^n$.
Dokazovali bomo v dveh delih. Najprej bomo dokazali zveznost funkcije $F$ na $A^c$, potem pa še na $A$.
Izberimo neko poljubno majhno pozitivno realno število $\varepsilon > 0$ in $x_0 \in A^c$. Potem obstaja tako pozitivno realno število $r>0$, da je odprta krogla s polmerom $r$ vsebovana v $A^c$, torej $B(x_0, r) \subset A^c$. Na $B(x_0, r)$ ja funkcija $D_a(x) := \frac{d(x, a)}{d(x, A)}$ zvezna za vsak parameter $a$, saj sta funkciji $d(x, a)$ in $d(x, A)$ zvezni in različni od $0$. Torej obstaja realno število $\delta \in (0, r)$, da je $|D_a(x_0) - D_a(x)| < \varepsilon$ za vsak $x \in B(x_0, \delta)$.
Denimo, da je infimum $F(x_0)$ dosežen pri $a_0$, torej je $F(x_0) = f(a_0) + D_{a_0}(x_0) - 1$. Za vrednosti $F(x)$ lahko naredimo naslednje ocene:
\begin{equation*} \label{eq1}
\begin{split}
F(x) & = F(x) - F(x_0) + F(x_0) = \\
& = \inf \left \{ f(a) + D_a(x) - 1; a \in A \right \} - F(x_0) + F(x_0) \leq \\
& \leq (f(a_0) + D_{a_0}(x) - 1) -  (f(a_0) + D_{a_0}(x_0) - 1) + F(x_0) = \\
& = (D_{a_0}(x) -  D_{a_0}(x_0)) + F(x_0) \leq \\
& \leq \varepsilon + F(x_0), \\
& \implies F(x_0) - F(x) \geq -\varepsilon.
\end{split}
\end{equation*}

Podobno ocenimo:
\begin{equation*} \label{eq1}
\begin{split}
F(x_0) & = F(x_0) - F(x) + F(x) = \\
& = F(x_0) - \inf \left \{ f(a) + D_a(x) - 1; a \in A \right \} + F(x) \leq \\
& \leq (f(a_0) + D_{a_0}(x_0) - 1) -  (f(a_0) + D_{a_0}(x) - 1) + F(x) = \\
& = (D_{a_0}(x_0) -  D_{a_0}(x)) + F(x) \leq \\
& \leq \varepsilon + F(x), \\
& \implies F(x_0) - F(x) \leq \varepsilon.
\end{split}
\end{equation*}
Iz zgornjih ocen ugotovimo, da je $|F(x_0) - F(x)| \leq \varepsilon$, kar zagotavlja zveznost funkcije $F$ na $A^c$.

Da bi preverili zveznost $F$ tudi na množici $A$ izberemo $x_0 \in A$ in $\varepsilon > 0$. Ker je funkcija $f$ zvezna na $A$ obstaja tak $\delta > 0$, da je $|f(x_0) - f(x)| < \varepsilon$ za vsak $x \in A$, za katerega velja $d(x_0, x) < \delta$. Zaradi lažjega ocenjevana naredimo naslednje premisleke. Ker je funkcija $f$ definirana na kompaktni množici obstaja realno število $M$, da je $f(a) \leq M$ za vsak $a \in A$. Določimo tak $\mu >0$, pri katerem je funkcija $D_a(x) \geq M + 1$ za vse $a \in A \setminus B(x_0, \delta)$ in vse $x \in B(x_0, \mu)$. Naj bo $a \in B(x_0, \delta)^c$, potem je
$D_a(x)|_{B(x_0, \mu)}= \frac{d(x, a)}{d(x, A)} \geq \frac{\delta - \mu}{\mu}$. Torej moramo določiti $\mu > 0$, da bo veljala neenačba $\frac{\delta - \mu}{\mu} > M + 1$. Izračunamo $\mu <\frac{\delta}{M + 2}$.


\begin{figure}[h!]
	\centering
	\begin{tikzpicture}
		\draw[black] plot [smooth cycle, tension = 0.8] coordinates {(-4, 2)  (3.5, 2) (3.8, -0.5) (0.7, -0.7) (0.6, 1.4) (-0.5, 1.2)  (-0.3, -0.7) (-3, -0.8)};
%		\filldraw[black] (-4, 2) circle (1.5pt);
%		\filldraw[black] (3.5, 2) circle (1.5pt);
%		\filldraw[black] (3.8, -0.5) circle (1.5pt);
%		\filldraw[black] (0.7, -0.7) circle (1.5pt);
%		\filldraw[black] (0.6, 1.4) circle (1.5pt);
%		\filldraw[black] (-1, -0.5) circle (1.5pt);
%		\filldraw[black] (-2.8, -1) circle (1.5pt);
		\filldraw[black] (0.65, 0.2) circle (1.5pt) node[black,right] {$x_0$};
		\filldraw[black] (0, 0.3) circle (1.5pt) node[black, right] {$x$};
		\filldraw[black] (-0.25, 0.2) circle (1.5pt) node[black, left] {$b$};
		\filldraw[black] (1.2, 1.1) circle (1.5pt) node[black, right] {$c$};
		\draw[black, style = dashed] (0.65, 0.2) circle (22pt);
		\draw[black, style = dashed] (0.65, 0.2) circle (50pt);
		\draw[black, <-] (2.2, 1.2) -- (4.7, 1.55);
		\draw[black, <-] (1.5, 0.2) -- (4.7, 0.22);
		\node at (5.5, 1.55){$B(x_0, \delta)$};
		\node at (5.5, 0.1){$B(x_0, \mu)$};
	\end{tikzpicture}
	\caption{Pri dokazovanju zveznosti funkcije $F$ v točki $x_0$ želimo poiskati tako pozitivno realno število $\mu$, da se vrednosti funkcije $F$ za vsako točko $x \in B(x_0, \mu)$ od vrednosti funkcije $F(x_0)$ razlikujejo največ za v naprej predpisani $\varepsilon$.}\label{fig:zveznarazsiritev}
\end{figure}

Sedaj imamo dve možnosti. Če je $x \in A \cap B(x_0, \mu)$, potem je $|f(x_0) - f(x)| < \varepsilon$. Če pa je $x \in A^c \cap B(x_0, \mu)$ vemo, da obstajata točki $b \in \partial A$, za katero je $d(x, b) = d(x, A)$ in $c \in A $ z lastnostjo $F(x) = f(c) + D_c(x) -1$. Iz izbire števila $\mu$ je jasno, da ležita točki $b, c \in B(x_0, \delta)$ (slika~\ref{fig:zveznarazsiritev}). Lahko naredimo podobne ocene, kot prej.
 \begin{equation*} \label{eq1}
\begin{split}
F(x) & = \inf_{a \in A} \{ f(a) + D_a(x) - 1\} \leq f(b) \leq f(x_0) +\varepsilon = F(x_0) +\varepsilon\\
F(x) & = f(c) + D_c(x) -1 \geq f(c) \geq f(x_0) - \varepsilon =  F(x_0) -\varepsilon \\
\end{split}
\end{equation*}
 Torej velja $| F(x_0) - F(x) | \leq \varepsilon$, kar zaključi dokaz.
\end{dokaz}

Lemo~\ref{lem:razsiritev} enostavno posplošimo tudi na preslikave, ki slikajo v večrazsežni evklidski prostor.

\begin{posledica}
Naj bo $A$ kompaktna podmnožica evklidskega prostora $\R^n$ in naj bo \mbox{$f : A \to \R^n$} zvezna preslikava. Potem obstaja zvezna preslikava $F : \R^n \to \R^n$, da je za vsak $x \in A$, $F(x) = f(x)$.
\end{posledica}

\begin{dokaz}
Imamo zvezno preslikavo $f : A \to \R^n$, kjer je $f(x) = (f_1(x), f_2(x), \dots , f_n(x))$ zapis preslikave $f$ po komponentah. Vse komponentne funkcije $f_i : A \to \R$ zadoščajo pogojem leme~\ref{lem:razsiritev}, zato jih lahko razširimo do funkcij $F_i : \R^n \to \R$. Če definiramo preslikavo $F : \R^n \to \R^n$ s predpisom $F(x) = (F_1(x), F_2(x), \dots , F_n(x))$, dobimo zvezno razširitev preslikave $f$.
\end{dokaz}
Sedaj znamo razširiti zvezno preslikavo iz kompaktne množice na cel prostor $\R^n$. Včasih pa si poleg spremembe definicijskega območja želimo imeti nekaj vpliva tudi na zalogo vrednosti preslikave. Želeli bi si, da se zaloga vrednosti pri razširitvi čim manj spremeni. Naivno bi lahko želeli, da ta ostane celo enaka, a to ni vedno mogoče, kar pokaže primer~\ref{protiprimer}.

\begin{primer}\label{protiprimer}
Poglejmo si identično preslikavo $f : \partial \I^n \to \I^n \setminus \{ \0 \}$. Po izreku \ref{izr:PM} ne obstaja zvezna razširitev $F : \I^n \to \I^n \setminus \{ \0 \}$, saj mora obstajati $x \in \I^n$, da je $F(x) = \0$. 
\end{primer}

V posebnih primerih obstaja taka razširitev preslikave, ki se izogne določeni točki. Brez izgube splošnosti bomo pokazali, da obstajajo pogoji, ki nam zagotavljajo zvezno razširitev preslikave, ki noben element definicijskega območja ne slika v $\0$. Preden navedemo pogoje, pa si poglejmo nekaj lastnosti preslikav, ki so ključne pri dokazu.

\begin{trditev}[\protect{\cite{zvjeenakzv}}]\label{trd:zvjeenakzv}
Zvezna funkcija definirana na kompaktni množici je enakomerno zvezna.
\end{trditev}
\begin{dokaz}
Predpostavimo, da je $K \in \R^n$ poljubna kompaktna množica in $f : K \to \R^n$ zvezna preslikava. Pokazati želimo, da obstaja tak $\delta > 0$, da za vsak še tako majhen $\varepsilon > 0$ in vsaka $x, y \in K$, za katera je $d(x, y) < \delta$, velja $d(f(x), f(y)) < \varepsilon$. Ker je $f$ zvezna funkcija za vsako točko $z \in K$ obstaja tako pozitivno realno število $\delta_z$, da je $f(B(z, \delta_z)) \subset B(f(z), \frac{\varepsilon}{4})$. Družina množic $\pU = \left \{ B(z, \delta_z); z \in K \right \}$ tvori odprto pokritje prostora $K$. Zaradi kompaktnosti $K$ lahko izberemo končno poddružino $\pA =  \left \{ (B(x_1, \delta_{x_1}), \dots, B(x_n, \delta_{x_n}) \right \}$, ki je še vedno pokritje množice $K$. Definiramo število $\delta = \min_i \delta_{x_i}$. Za poljubni števili $x, y \in K$, ki zadoščata pogoju sta med seboj oddaljeni manj kot $\delta$, imamo dve možnosti. Če obstaja taka krogla $B(x_i, \delta)$, da je $x, y \in B(x_i, \delta)$, potem sta $f(x), f(y) \in B(f(x_i), \frac{\varepsilon}{4})$ in je $d(f(x), f(y)) < \frac{\varepsilon}{2}$. V nasprotnem primeru obstajata krogli  $B(x_i, \delta)$ in $B(x_j, \delta)$ z nepraznim presekom, za kateri velja $x \in B(x_i, \delta)$ in $y \in B(x_j, \delta)$. V tem primeru je $f(x) \in B(f(x_i), \frac{\varepsilon}{4})$ in $f(y) \in B(f(x_j), \frac{\varepsilon}{4})$, kjer je $B(f(x_i), \frac{\varepsilon}{4}) \cap B(f(x_j), \frac{\varepsilon}{4}) \neq \emptyset$. Zato zagotovo velja $d(f(x), f(y)) < \varepsilon$, torej je funkcija res enakomerno zvezna.
\end{dokaz}
Poglejmo sedaj, kakšne pogoje potrebujemo, da lahko funkcijo razširimo tako, da se izognemo ničli.
%Mi pa si bomo v nadaljevanju želeli poiskati ravno take funkcije, ki jih lahko razširimo tako, da je pri tem izognemo ničli. 
%\begin{definicija}
%Družini $\mathcal{A}$ podmnožic množice $X$, za katero velja $\bigcup_{A \in \mathcal{A}} = X$ pravimo \emph{pokritje množice} $X$. Če družina $\mathcal{A}$ vsebuje zgolj odprte množice, jo imenujemo \emph{odprto pokritje množice} $X$.
%\end{definicija}

%\begin{lema}[Lebesgueova lema]\label{lem:lebesgue}
%Za vsako odprto pokritje $\pU$ kompaktnega metričnega prostora $X$ obstaja pozitivno realno število $\lambda$, ki ga imenujemo Lebesgueovo število, tako da je vsaka podmnožica prostora $X$ z diametrom manjšim od $\lambda$ vsebovana v neki množici $U \in \pU$
%\end{lema}

%\begin{dokaz}
%Dokazovali bomo s protislovjem. Predpostavimo, da Lebesgueovo število ne obstaja. Potem lahko za vsako naravno število $n \in \N$, izberemo $x_n \in X$, da krogla $B(x_n, \frac{1}{n})$ ne leži v nobeni množci $U \in \pU$ tako oprto pokritje $\pU$ množice $X$, da za vsak $\delta >0$  obstaja $x$, da nobena množica $U \in \pU$ ne vsebuje krogle $B(x, \delta)$. Torej za vsak $n \in \N$ lahko izberemo $x_n \in X$, da noben $U \in \pU$ ne vsebuje krogle $B(xn, \frac{1}{n})$. Ker je $X$ kompakten obstaja konvergentno podzaporedje $x_{n_k}$, ki konvergira k $y \in X$.
%\end{dokaz}



\begin{lema}[razširitev, ki se izogne ničli \protect{\cite[str.\ 133, Proof of Lemma]{Kulpa}}]\label{lem:razsiritev-nic}
Naj bo $X$ kompaktna podmnožica evklidskega prostora $\R^n$ in naj bo \mbox{$f : X \to \R^n \setminus \left \{ \0 \right \}$}. Potem za vsak $\varepsilon > 0$ in za vsako kompaktno množico s prazno notranjostjo $Y \subset \R^n$ obstaja zvezna preslikava $F : X \cup Y \to \R^n \setminus \{ \0 \}$, da velja: $\| F(x)-f(x) \| < \varepsilon$  za vsak $x \in X$
\end{lema}

\begin{dokaz}
Določimo poljubno število $\varepsilon > 0$ in množici $X$ in $Y$, kot v lemi \ref{lem:razsiritev-nic}. Ker je množica $X \cup Y \subset R^n$ omejena, obstaja realno število $a > 0$, da je unija množic $X \cup Y$ vsebovana v kocki $\I^n = \left [ -a, a \right ]^n$. Po lemi \ref{lem:razsiritev} lahko vsako zvezno preslikavo $f : X \to \R^n \setminus \{ \0 \}$ razširimo do zvezne preslikave $g : \I^n \to \R^n$. Izberemo tako realno število $\delta \in (0, \frac{\varepsilon}{2})$, da je $f(X) \cap B( \0, 2\delta) = \emptyset$. Ker je funkcija $g$ zvezna in definirana na kompaktni množici $\I^n$, je po trditvi~\ref{trd:zvjeenakzv} enakomerno zvezna. Zato obstaja realno število $\mu$, da za vsako množico $A$, katere diameter $\diam(A)$ je manjši od $\mu$, velja $\diam(g(A)) < \delta$. Naj bo $k \in \N$ dovolj veliko naravno število, da za vsak simpleks $S$ iz $k$-baricentrične triangulacije $T_k$ kocke $\I^n$ velja $\diam(S) < \mu$. Če označimo množico vozlišč v triangulaciji $T_k$ z $V$, nam preslikava $g|_V : V \to \R^n$ enolično določa zvezno preslikavo $h : \I^n \to \R^n$, ki vsako točko $x \in S = \left < z_0, z_1, \dots, z_n \right > \in T_k$ z baricentričnimi koordinatami $x = (t_0, t_1, \dots, t_n)_b$ slika v 
$$h(x) = \sum_{i=0}^n t_i g(z_i).$$
Zaradi izbire triangulacije $T_k$ je za vsak simpleks $S = \langle  z_1, z_2, \dots, z_n \rangle \in T_k$ diameter $\diam(S)$ manjši od $\mu$, torej zaradi enakomerne zveznosti funkcije $g$ velja $\diam (g(S)) < \delta$. Torej obstaja krogla $B$ s premerom $\delta$, ki vsebuje množico $g(S)$. Funkcijo $h$ smo konstruirali tako, da je množica $h(S)$ konveksna ogrinjača točk $g(z_0), g(z_1), \dots, g(z_n)$. Ker je krogla $B$ konveksna množica, ki vsebuje točke $g(z_0), g(z_1), \dots, g(z_n)$, vsebuje tudi njihovo konveksno ogrinjačo $h(S)$. Sklep sledi iz dejstva, da je konveksna ogrinjača danih točk najmanjša konveksna množica, ki vsebuje dane točke. Iz tega ugotovimo, da je $\| g(x) - h(x) \| < \delta$ za vsak $x \in \I^n$. Ker je funkcija $g$ razširitev funkcije $f$, sklepamo, da velja $\| f(x) - h(x) \| < \delta$ za vsak $x \in X$.

Sedaj opazujmo množico točk $\left \{ g(z_0), g(z_1), \dots, g(z_n) \right \}$. Če je to afino odvisna množica, potem je množica $h(S)$ določena z $n$ med seboj odvisnimi vektorji, torej leži v nekem $(n-1)$-dimenzionalne podprostoru prostora $\R^n$ in ima kot taka prazno notranjost. Če pa je množica afino neodvisna, je množica $S \cap Y$ kompaktna množica s prazno notranjostjo, saj je presek množice $Y$, ki je kompaktna s prazno notranjostjo in množice $S$, ki je kompaktna. Zaradi zveznosti funkcije $h$, je tudi $h(S \cap Y)$ kompaktna množica s prazno notranjostjo. Res, zvezne funkcije ohranjajo kompaktnost. Denimo, da je notranjost množice $h(S \cap Y)$ neprazna. Potem obstaja neka odprta množica $V \subset \R^n$, ki je cela vsebovana v $h(S \cap Y)$. Ker je funkcija $h$ zvezna, je njena inverzna preslikava $h^{-1}$ odprta, zato je tudi množica $h^{-1}(V)$ odprta v prostoru $\R^n$. To pa ni mogoče, saj je množica $h^{-1}(V)$ vsebovana v množici $S \cap Y$, ki ima prazno notranjost, zato ima tudi množica $h^{-1}(V)$ prazno notranjost. Iz teh razmislekov ugotovimo, da je množica $h(Y) = \bigcup\limits_{S \in T_k} h(S \cap Y)$ kompaktna množica s prazno notranjostjo, saj jo dobimo s končno unijo kompaktnih množic s prazno notranjostjo. Ker je $f(X) \cap B(\0, 2 \delta) = \emptyset$, in ker je $\| f(x) - h(x) \| < \delta$ za vsak $x \in X$, je $h(X) \cap B(\0, \delta) = \emptyset$. Jasno je, da lahko izberemo točko $d \in B(\0, \delta) \setminus h(X \cup Y)$.

Definiramo preslikavo $F : X \cup Y \to \R^n$ s predpisom $F(x)  = h(x) - d$. Opazimo, da je $\| F(x) - f(x) \| \leq \| h(x) - f(x) \| + \| d \| < 2\delta < \varepsilon$, za vsak $x \in X$. Velja tudi $F(z) \neq \0$ za vsak $z \in X \cup Y$, saj bi enakost $F(z) = 0$ implicirala enakost $h(z) = d$, kar pa nasprotuje predpostavki, da je $d \notin h(X \cup Y)$. Preslikava $F$ je res iskana razširitev.
\end{dokaz}

%####################     5. POGLAVJE: IZREK O INVARIANCI ODPRTIH MNOŽIC     ####################
\section{Izrek o invarianci odprtih množic}\label{raz:ioiom}
V prejšnjih poglavjih smo si pripravili vse potrebno za dokaz izreka o invarianci odprtih množic, zato se bomo brez ovinkarjenja lotili dokaza. Nato pa bomo kot posledico izreka o invarianci odprtih množic dokazali še izrek o invarianci dimenzij. Pri tem bomo sledili~\cite{Kulpa}.
%
\begin{izrek}[Izrek o invarianci odprtih množic]\label{izr:main-theorem}
Naj bo $U \subset \R^n$ odprta podmnožica evklidskega prostora $\R^n$ in naj bo $h : U \rightarrow \R^n$ zvezna injektivna preslikava.
Potem je tudi slika $h(U)$ odprta množica v $\R^n$.
\end{izrek}
%
\begin{dokaz}
Naj bodo izpolnjene predpostavke izreka. Imamo množico U, ki je odprta podmnožica v $\R^n$ in zvezno injektivno preslikavo $h : U \rightarrow \R^n$. Izrek bo dokazan, če pokažemo, da je za vsak element $u$ iz množice $U$ točka $h(u)$ notranja točka za množico $h(U)$. Ker je $\R^n$ homogen prostor, take pa so tudi vse njegove odprte podmnožice, lahko predpostavimo, da je $u = 0_n$ in pokažemo, da je $h(0)$ notranja točka za $h(U)$. Izberimo tako pozitivno realno število $a > 0$, za katero je $\I^n \subset U$. Za dokaz izreka  je dovolj pokazati vsebovanost $b := h(0) \in \Int(h(\I^n))$. Od tu naprej bomo dokazovali s protislovjem. Privzeli bomo, da je $b \in \partial h(\I^n)$ in konstruirali funkcijo $f : \I^n \to \R^n \setminus \{ 0 \}$, tako da bo f zadoščala pogojem izreka~\ref{izr:PM}. To pa bo protislovje, saj mora taka funkcija po izreku~\ref{izr:PM} vsaj eno točko slikati v $0$. Na poti do protislovja si bomo seveda pomagali tudi z lemami, ki smo jih spoznali in dokazali v prejšnjih poglavjih. Predpostavimo torej, da je $b \in \partial (\I^n)$. Ker je $\I^n$ kompaktna podmnožica $\R^n$ in je $\R^n$ houssdorfov prostor, je funkcija $h|_{\I^n} : \I^n \to \R^n$ homeomorfizem. Zato obstaja tako pozitivno realno število $\delta > 0$, za katerega je $h^{-1}(B(b, 2 \delta)) \subset \Int(\I^n)$. Ker je $b \in \partial h(\I^n)$ je mogoče poiskati tak $c \in B(b, \delta) \setminus h(I^n)$. Enostavno se je prepričati, da je $b \in B(c, \delta)$ in $h^{-1} (B(c, \delta)) \subset \Int (\I^n)$.

% ###############        Slika dokaza izreka o invarianci odprtih množic      ############
\begin{figure}[h!]
	\centering
	\begin{tikzpicture}
% ###############          prva slika          ###############
		\filldraw[color=gray!18] plot [smooth cycle, tension = 0.9] coordinates {(-5.15, 3.35) (-5.15, 2.8) (-3.95, 2.7) (-4.25, 3.45)};
		\draw[gray, line width=0.5pt] plot [smooth cycle, tension = 0.9] coordinates {(-5.15, 3.35) (-5.15, 2.8) (-3.95, 2.7) (-4.25, 3.45)};
		\filldraw[color=gray!18, line width=1pt] plot [smooth cycle, tension = 1.1] coordinates {(-4.25, 2.15) (-4.7, 2.35) (-5.15, 2.25) (-4.7, 2.1)};
		\draw[gray, line width=0.5pt] plot [smooth cycle, tension = 1.1] coordinates {(-4.25, 2.15) (-4.7, 2.35) (-5.15, 2.25) (-4.7, 2.1)};
		\draw (-5.9, 1.8) rectangle (-3.3, 4.2);
		\filldraw[red] (-4.6, 3) circle (1.5pt) node[black, above right=-0.5mm] {$0$};
		\draw (-4.2, 3.1) ;	
		 \draw (-3.6, 3.85) node {$I^n$};
		
% ###############          druga slika          ###############
		\begin{scope}
			\clip plot [smooth cycle, tension = 0.3] coordinates {(2.3, 1.45) (5.2, 1.5) (5.25, 4.65) (4.25, 4.7) (4.2, 2.5) (3.15, 2.5) (3.7, 4.2) (3.5, 4.4) (2.8,3.7)  (2.5, 3.2) (2.3, 2.5)};
			\clip (3.9, 4) circle (21pt);
			\fill[color=gray!18] (-2,1.5) rectangle (6,5);
		\end{scope}
		\draw plot [smooth cycle, tension = 0.3] coordinates {(2.3, 1.45) (5.2, 1.5) (5.25, 4.65) (4.25, 4.7) (4.2, 2.5) (3.15, 2.5) (3.7, 4.2) (3.5, 4.4) (2.8,3.7)  (2.5, 3.2) (2.3, 2.5)};
		\filldraw[black] (3.9, 4) circle (2pt) node[black, below right=-0.4mm]{$c$};
		\filldraw[red] (4.21, 4.2) circle (1.5pt) node[black, right=-0.4mm] {$b$};
		\draw[black] (3.9, 4) circle (21pt);
		\draw (4.6, 2) node {$h(I^n)$};
	
% ###############          tretja slika          ###############
		\draw[lightgray, line width=0.5pt] (3.9, -2) circle (21pt);
		\draw[black, line width=1pt] plot [smooth cycle, tension = 0.3] coordinates {(2.3, -4.55) (5.2, -4.5) (5.25, -1.35) (4.25, -1.3) (4.2, -3.5) (3.15, -3.5) (3.7, -1.8) (3.5, -1.6) (2.8, -2.3)  (2.5, -2.8) (2.3, -3.5)};
		\filldraw[white] (3.9, -2) circle (20.5pt);
		\begin{scope}
			\clip (3.9, -2) circle (21pt);
			\draw[lightgray, line width=0.5pt] plot [smooth cycle, tension = 0.3] coordinates {(2.3, -4.55) (5.2, -4.5) (5.25, -1.35) (4.25, -1.3) (4.2, -3.5) (3.15, -3.5) (3.7, -1.8) (3.5, -1.6) (2.8, -2.3)  (2.5, -2.8) (2.3, -3.5)};
		\end{scope}
		\begin{scope}
			\clip (3.9, -2) -- (3.4, -1.26) -- (2.5, -1) -- (3.35, -2.7) -- cycle;
			\draw[black, line width=1pt] (3.9, -2) circle (21pt);
		\end{scope}
		\begin{scope}
			\clip plot [smooth cycle, tension = 0.3] coordinates {(2.3, -4.55) (5.2, -4.5) (5.25, -1.35) (4.25, -1.3) (4.2, -3.5) (3.15, -3.5) (3.7, -1.8) (3.5, -1.6) (2.8, -2.3)  (2.5, -2.8) (2.3, -3.5)};
			\draw[black, line width=1pt] (3.9, -2) circle (21pt);
		\end{scope}
		\filldraw[black] (3.9, -2) circle (2pt) node[black, below right=-0.4mm]{$c$};
		\draw[black, line width=0.5pt, -Stealth] ($($(3.9, -2)!.5!(4.525, -1.6)$)!5pt!(3.9, -2)$) --  ($($(3.9, -2)!.5!(4.525, -1.6)$)!6pt!(4.525, -1.6)$);
		\draw[black, line width=0.5pt, -Stealth] ($($(3.9, -2)!.5!((3.9, -2.75)$)!5pt!(3.9, -2)$) --  ($($(3.9, -2)!.5!((3.9, -2.75)$)!6pt!((3.9, -2.75)$);
		\draw[black, line width=0.5pt, -Stealth] ($($(3.9, -2)!.5!(3.3, -1.6)$)!5pt!(3.9, -2)$) --  ($($(3.9, -2)!.5!(3.3, -1.6)$)!6pt!(3.3, -1.6)$);
		\filldraw[red] (4.525, -1.6) circle (1.5pt) node[black, right=-0.3mm] {$l(b)$};
		\draw (3.1, -1.1) node {$Y$};
		\draw (4.8, -4) node {$X$}; 
		
	% ###############          četrta slika          ###############
		\draw (-5.9, -4.2) rectangle (-3.3, -1.8);
		\draw[black, line width=1pt] plot [smooth cycle, tension = 1] coordinates {(-4.5, -2.7) (-4.3, -3.1) (-4.7, -3.25) (-5.1, -2.8)};
		\draw[black, line width=1pt] plot [smooth, tension = 1] coordinates {(-5.2, -3.45) (-4.9, -3.38) (-4.6, -3.55) (-4.25, -3.5)};
		\filldraw[red] (-4.72, -3.5) circle (1.5pt) node[black, below] {$f(0)$};
		\filldraw[black] (-4.6, -3) circle (2pt);
		\draw[black, line width=1pt] plot [smooth cycle, tension = 0.9] coordinates { (-3.1, -1.7) (-4.2, -1.7) (-5.9, -1.6) (-5.75, -2.8) (-5.85, -4) (-4.3, -4.4) (-3.15, -3.7) (-3.1, -2.8)};
	
	% ###############          puščice          ###############
		\draw [black, line width=1.2pt, -Stealth] (-1.7,3) -- (1.2,3);
		\draw (-0.4, 3.6) node {$h$};
		\draw [black,  line width=1.2pt, -Stealth] (1.2, -3) -- (-1.7, -3);
		\draw (0, -2.7) node {$g$};
		\draw [black,  line width=1.2pt, -Stealth] (-4.6, 1) -- (-4.6, -0.8);
		\draw (-3.1, 0.2) node {$f  := g \circ l \circ h$};
		\draw [black,  line width=1.2pt, -Stealth] (3.9, 1) -- (3.9, -0.8);
		\draw (4.5, 0.2) node {$l$};  
	\end{tikzpicture}
	\caption{Skica dokaza izreka~\ref{izr:main-theorem}.}
\end{figure}

Označimo $X := h(\I^n) \setminus B(c, \delta)$ in $Y := \partial B(c, \delta)$. Definiramo zvezno preslikavo $l : h(\I^n) \cup Y \to X \cup Y$ s predpisom:
\[  l(x) = \left\{
\begin{array}{ll}
	c + \frac{x - c}{\| x - c \|} \cdot \delta &, x \in h(\I^n) \cup B(c, \delta) \\
	x &, x \in X. \\
\end{array} 
\right. \]
S pomočjo leme~\ref{lem:razsiritev-nic} lahko preslikavo $h|_X : X \to \R^n \setminus \{ 0 \}$ razširimo do zvezne preslikave $g : X \cup Y \to \R^n \setminus \{ 0 \}$, za katero za vsak $x \in X$ velja $\| g(x) - h^{-1}(x) \| < a$.
Sedaj lahko definiramo preslikavo $f = (f_1, f_2, \dots, f_n) : \I^n \to \R \setminus \{ 0 \}$ kot kompozitum $f := g \circ l \circ h$. Ker ta funkcija slika iz $\I^n$ in ne zavzame ničle, je za protislovje dovolj, če se prepričamo, da funkcija ustreza pogojem izreka~\ref{izr:PM}. Vzemimo $t \in \I_i^-$. Velja $l(h(t)) = h(t)$, saj je $h(t) \in X$. Za normo vektorja $f(t) - t$ lahko naredimo naslednje ocene:
$$\| f(t) - t \| = \| g(l(h(t))) - h{-1}(h(t)) \| = \| g(h(t) - h{-1}(h(t)) \| < a.$$ 
Ker je $t_i = - a$ je $| f_i (t) - t_i | = | f_i (t) - ( - a) | \leq | f (t) - t | < a$ torej je $f_i(t) < 0$. Podobno lahko tudi v primeru, ko je $t_i \in \I_i^+$  sklepamo, da je $f_i(t) > 0$. Ugotovili smo, da je  $f_i(\I_i^-) < 0$ in $f_i(\I_i^+) > 0$, zato bi po predpostavkah izreka~\ref{izr:PM} moral obstajati $x \in \I^n$, ki se z $f$ slika v $0$. To je protislovje, torej je $b \in \Int (h(\I^n)$ in je $h(U)$ odprta podmnožica v $\R^n$.
\end{dokaz}
Izrek je bil zelo pomemben, tudi zaradi spodnje posledice.
S pomočjo izreka~\ref{izr:main-theorem} lahko enostavno izpolnimo obljubo iz uvoda in dokažemo izrek o invarianci dimenzij, ki razreši vsaj nekatere nejasnosti, ki se nanašajo na pojem dimenzije.
\begin{posledica}[Izrek o invarianci dimenzije]\label{izr:dim_izr}
Naj bosta $m$ in $n$ naravni števili, potem sta Evklidska prostora $\R^m$ in $\R^n$ homeomorfna, če in samo če je $m = n$.
\end{posledica}

\begin{dokaz}
Denimo, da sta za neki dve naravni števili $m$ in $n$ Evklidska prostora $\R^m$ in $\R^n$ homeomorfna. Torej obstaja zvezna bijektivna preslikava $f : \R^n \to \R^m$ z zveznim inverzom $f^{-1} : \R^m \to \R^n$. Dokazovali bomo s protislovjem. Predpostavimo, da je $m \neq n$, brez izgube splošnosti lahko predpostavimo , da je $m < n$. Označimo z $i$ vložitev, torej preslikavo iz prostora $\R^m$ v $\R^n$, ki je definirana s predpisom $i(x_1, \dots, x_m) = (x_1, \dots, x_m, 0, \dots, 0) \ in \R^n$. Tedaj je preslikava $h : \R^n \to \R^n$ definirana kot kompozitum $h := i \circ f$ zvezna injektivna preslikava, zato je po izreku~\ref{izr:main-theorem} odprta. Toda slika prostora $\R^n$, ki je odprta podmnožica same sebe, s funkcijo $h$ je množica 
$$\left \{ (x_1, x_2, \dots, x_m, 0, \dots, 0) \in \R^n ; \text{ kjer so } x_i \in \R \text{ za vsak } i \in \{1, \dots, m \}  \right \},$$
 ki pa je zaprta podmnožica prostora $\R^n$. Torej mora biti res $m = n$.
\end{dokaz}

\section*{Slovar strokovnih izrazov}

\geslo{continuous}{zvezen}
\geslo{uniformly continuous}{enakomerno zvezen}
\geslo{convex}{konveksen}
\geslo{convex hull}{konveksna ogrinjača}
\geslo{compact}{kompakten -- metrični prostor je kompakten, če vsako odprto pokritje prostora vsebuje končno podpokritje; podmnožica evklidskega prostora je kompaktna natanko tedaj, ko je omejena in zaprta}
\geslo{cover}{pokritje}
\geslo{cube}{kocka}
\geslo{dimension}{dimenzija}
\geslo{extention}{razširitev}
\geslo{face}{lice}
\geslo{homeomorphism}{homeomoefizem -- preslikava je homeomorfizem, če je zvezna in ima zvezen inverz}
\geslo{homeomorphic}{homeomoefen -- prostora $X$ je homeomorfen prostoru $Y$, če obstaja homeomorfizem $f : X \to Y$}
\geslo{sequence}{zaporedje}
\geslo{simplex}{simpleks}
\geslo{triangulation}{triangulacija}


\geslo{}{}

%seznam uporabljene literature
\phantomsection                            % da prav delujejo hiperlinki
%\addcontentsline{toc}{section}{\bibname}   % dodajmo v kazalo
\bibliographystyle{fmf-sl}                 % uporabljen stil je v datoteki fmf-sl.bst, na voljo tudi angleška verzija
\bibliography{\literatura}        

\end{document}

